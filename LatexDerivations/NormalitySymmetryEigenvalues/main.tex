\documentclass[11pt]{article}
\usepackage[top=40mm,bottom=40mm,left=20mm,right=20mm]{geometry}

\usepackage[utf8]{inputenc}
\usepackage{parskip}
\usepackage{amsmath}
\usepackage{physics}
\usepackage{stackengine}
\usepackage{float}
\usepackage{cleveref}
\usepackage{graphics}
\usepackage{siunitx}

\allowdisplaybreaks

\newcommand\textbff[1]{\textbf{\boldmath #1}}
\newcommand\circled[1]{\raisebox{.5pt}{\textcircled{\raisebox{-.9pt} {#1}}}
    }
\newcommand{\shortnote}[1]{\textit{\footnotesize (#1)}}

\newcommand{\pp}{\partial}
\newcommand{\mgrad}{\su{\nabla}}

% We often use underlines here
\newcommand\barbelow[1]{\stackunder[1.2pt]{$#1$}{\rule{1.5ex}{.1ex}}}
\newcommand{\su}[1]{\barbelow{#1}}
\newcommand{\du}[1]{\barbelow{\barbelow{#1}}}

\newcommand{\EE}{$\du{E}$}
\newcommand{\PP}{$\du{\Pi}$}
\newcommand{\ddelta}[4]{\delta_{#1#3}\delta_{#2#4} + \delta_{#1#4}\delta_{#2#3}}

\newcommand{\YY}[3][j]{E_{#2#1}E_{#3#1}^*}

\def\onedot{$\mathsurround0pt\ldotp$}
\def\cddot{\mathbin{
    \vcenter{\baselineskip1ex \vspace{-0.1ex}\hbox{\onedot}\hbox{\onedot}}
}}


\begin{document}
\begin{center}
    \LARGE
    \textbf{Showing that requiring normality, symmetry and eigenvalues gives a form for $\du{E}$}
\end{center}
\vspace{1em}

As motivated in Jack's thesis, if we \textbff{take the eigenvalues of $\du{E}$ to be proportional to $\psi$, add to zero} (to make $\du{E}$ traceless) \textbff{and all except one be equal} (nearly fully degenerate) to choose one "special" axis.
Then they must all be $\frac{-\psi}{d}$ except the special one which is $\frac{(d-1)\psi}{d}$ (up to scaling).
If we further \textbff{require $\du{E}$ to be normal}, then it can be diagonalised using a unitary matrix (this I got from wikipedia), such that
\begin{align}
    \du{E} &= \du{U}^\dagger \begin{pmatrix}
        \frac{-\psi}{d} & \dots & 0 \\
        \vdots & \ddots & \vdots\\
        0 & \dots & \frac{(d-1)\psi}{d}
    \end{pmatrix} \du{U} \qq{where $\du{U}$ is unitary} \\
    &= \frac{\psi}{d} \du{U}^\dagger \begin{pmatrix}
        -1 & \dots & 0 \\
        \vdots & \ddots & \vdots\\
        0 & \dots & d-1
    \end{pmatrix} \du{U} = \frac{\psi}{d} \du{U}^\dagger \qty(\begin{pmatrix}
        0 & \dots & 0 \\
        \vdots & \ddots & \vdots\\
        0 & \dots & d
    \end{pmatrix} - \du{\delta}) \du{U} \\
    &= \frac{\psi}{d} \qty(\du{U}^\dagger \begin{pmatrix}
        0 & \dots & 0 \\
        \vdots & \ddots & \vdots\\
        0 & \dots & d
    \end{pmatrix} \du{U} - \du{\delta}) \\
\end{align}
now switching to index notation
\begin{align}
    E_{ij} &= \psi \qty(U^\dagger_{id}U_{dj} - \frac{\delta_{ij}}{d}) \qq{$d$ is the dimension here, not a dummy index} \\
    &= \psi \qty(U^*_{di}U_{dj} - \frac{\delta_{ij}}{d})
\end{align}
So only a single (the last) row of $\du{U}$ has any effect.
To move further, recall that the \textbff{rows (or columns) or a unitary matrix for a complex orthonormal basis}.
Thus, if we are only considering a single row, the only constraint that applies is that it's nor has to be 1.
I also switch to \textbff{specifically considering $d=3$} here to make it simpler, but I expect it's general.
Parametrize this last row as follows
\begin{align}
    \su{U_3} &\leftrightarrow \begin{pmatrix} R_ae^{i\phi_a} & R_be^{i\phi_b} & R_ce^{i\phi_c} \end{pmatrix} \qq{with} R_a, R_b, R_c \geq 0 \qq{and} R_a^2 + R_b^2 + R_c^2 = 1 \\
    &\qq{and} \nonumber \\
    \su{U^*_3} &\leftrightarrow \begin{pmatrix} R_ae^{-i\phi_a} & R_be^{-i\phi_b} & R_ce^{-i\phi_c} \end{pmatrix}
\end{align}
this gives $\du{E}$ as
\begin{align}
    \du{E} &\leftrightarrow \psi \qty(\begin{pmatrix}
        R_a^2 & R_a R_b e^{i(\phi_b-\phi_a)} & R_a R_c e^{i(\phi_c-\phi_a)} \\
        R_a R_b e^{-i(\phi_b-\phi_a)} & R_b^2 & R_b R_c e^{i(\phi_c-\phi_b)} \\
        R_a R_c e^{-i(\phi_c-\phi_a)} & R_b R_c e^{-i(\phi_c-\phi_b)} & R_c^2
    \end{pmatrix} - \frac{\du{\delta}}{d}) \label{eq:presymm}
\end{align}

\pagebreak
Finally, \textbff{requiring $\du{E}$ to be symmetric} means each of the phase differences above must be an integer multiple of $\pi$ (as $e^{i\theta}=e^{-i\theta}$ iff $\theta$ is an integer multiple of $\pi$).
Using $\phi_b-\phi_a = k\pi$ and $\phi_c-\phi_a=n\pi$ gives $\phi_c-\phi_b=(n-k)\pi$.
Focusing on the matrix in \cref{eq:presymm}, noting that the terms in it only depend on whether each of $n$, $k$, $n-k$ are odd or even we end up with 4 options for $\du{E}$, each equivalent to $\psi\qty(\su{R}\su{R}-\frac{\du{\delta}}{d})$ with a different $\su{R}$.

\begin{table}[H]
    \begin{center}
        \begin{tabular}{ c|c|c c c c } 
            $k$ & $n$ & $n-k$ & & The term appearing in $\du{E}$ & $\su{R}$ \\
             \hline
             even & even & even & $\leftrightarrow$ & $\begin{pmatrix}R_a^2 & R_a R_b & R_a R_c \\ R_a R_b & R_b^2 & R_b R_c \\ R_a R_c & R_b R_c & R_c^2 \end{pmatrix}$ & $\begin{pmatrix}R_a & R_b & R_c\end{pmatrix}$ \\
            &&&&\\
             even & odd & odd & $\leftrightarrow$ & $\begin{pmatrix}
                R_a^2 & R_a R_b & -R_a R_c \\
                R_a R_b & R_b^2 & -R_b R_c \\
                -R_a R_c & -R_b R_c & R_c^2
                 \end{pmatrix}$ & $\begin{pmatrix}R_a & R_b & -R_c\end{pmatrix}$ \\
            &&&&\\
             odd & even & odd & $\leftrightarrow$ & $\begin{pmatrix}
                R_a^2 & -R_a R_b & R_a R_c \\
                -R_a R_b & R_b^2 & -R_b R_c \\
                R_a R_c & -R_b R_c & R_c^2
                 \end{pmatrix}$ & $\begin{pmatrix}R_a & -R_b & R_c\end{pmatrix}$ \\
            &&&&\\
             odd & odd & even & $\leftrightarrow$ & $\begin{pmatrix}
                R_a^2 & -R_a R_b & -R_a R_c \\
                -R_a R_b & R_b^2 & R_b R_c \\
                -R_a R_c & R_b R_c & R_c^2
                 \end{pmatrix}$ & $\begin{pmatrix}-R_a & R_b & R_c\end{pmatrix}$
        \end{tabular}
    \end{center}
    \caption{The 4 options for $\du{E}$}
\end{table}
Finally recalling that each of the $R_?$ components are positive and that $R_a^2 + R_b^2 + R_c^2 = 1$, we thus get that any $\du{E}$ must be of the form $\psi\qty(\su{N}\su{N} - \frac{\du{\delta}}{d})$ for some unit vector $\su{N}$ pointing into one of the 4 quadrants where at most 1 Cartesian component is negative, which spans exactly the rotational symmetry we require of the system.

\end{document}
