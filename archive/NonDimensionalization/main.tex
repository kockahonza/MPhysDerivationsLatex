\documentclass[11pt]{article}
\usepackage[top=40mm,bottom=40mm,left=20mm,right=20mm]{geometry}
\usepackage[utf8]{inputenc}
\usepackage{parskip}
\usepackage{amsmath}
\usepackage{physics}
\usepackage{stackengine}
\usepackage{float}
\usepackage{cleveref}
\usepackage{graphics}
\usepackage{siunitx}

\allowdisplaybreaks

\newcommand\textbff[1]{\textbf{\boldmath #1}}
\newcommand\circled[1]{\raisebox{.5pt}{\textcircled{\raisebox{-.9pt} {#1}}}
    }
\newcommand{\shortnote}[1]{\textit{\footnotesize (#1)}}

\newcommand{\pp}{\partial}
\newcommand{\mgrad}{\su{\nabla}}

% We often use underlines here
\newcommand\barbelow[1]{\stackunder[1.2pt]{$#1$}{\rule{1.5ex}{.1ex}}}
\newcommand{\su}[1]{\barbelow{#1}}
\newcommand{\du}[1]{\barbelow{\barbelow{#1}}}

\newcommand{\EE}{$\du{E}$}
\newcommand{\PP}{$\du{\Pi}$}
\newcommand{\ddelta}[4]{\delta_{#1#3}\delta_{#2#4} + \delta_{#1#4}\delta_{#2#3}}

\newcommand{\YY}[3][j]{E_{#2#1}E_{#3#1}^*}

\def\onedot{$\mathsurround0pt\ldotp$}
\def\cddot{\mathbin{
    \vcenter{\baselineskip1ex \vspace{-0.1ex}\hbox{\onedot}\hbox{\onedot}}
}}


\interdisplaylinepenalty=3000

\begin{document}
\begin{center}
    \LARGE
    \textbf{Non-dimensionalization}
\end{center}
\vspace{1em}
I will be using the following free energies here
\begin{align}
    f_\text{bulk} =&\enspace A |\du{E}|^2 + \frac{C}{2} |\du{E}|^4 \\
    f_\text{comp} =&\enspace b_1 |\su{\nabla}\du{E}|^2 \\
    f_\text{curv} =&\enspace b_2 |\nabla^2\du{E}|^2 \\
\end{align}
where $C$, and all the $b$s are positive, but $A$ can be negative.
And a time evolution of form
\begin{equation}
    \pdv{\du{E}}{t} = -\mu \fdv{F}{\du{E}^*}
\end{equation}
\noindent\fbox{\parbox{\textwidth}{
Notable differences from Jack's is that I omit the 2 extra factors of $\frac{1}{2}$ in the bulk contribution, deal with the $\parallel$ and $\perp$ parts slightly differently, and change $\mu$ to its inverse.
}}

Here 3 dimensions come up -- energy ($E$), length ($L$) and time ($T$) and the quantities above have units as follows:
\begin{center}
    \renewcommand{\arraystretch}{1.5}
    \begin{tabular}{c | c | c | c | c | c | c }
        quantity & \EE & $A$ & $C$ & all the $b_1$ & all the $b_2$ & $\mu$ \\
        \hline
        unit & 1 & $\frac{E}{L^3}$ & $\frac{E}{L^3}$ & $\frac{E}{L}$ & $E L$ & $\frac{1}{E T}$ \\
    \end{tabular}
\end{center}

This is all that is needed for 1c.a., and later on the projections operators can be introduced through $\su{\nabla} \rightarrow b^\parallel \du{\Pi} \cdot \su{\nabla} + b^\perp \du{T} \cdot \su{\nabla}$ with both $b$s in there being dimensionless.
Now there is currently more degrees of freedom here than needed, as both the $b_?$ and the $b^?$ affect the overall magnitudes of the terms in the free energy.
I'm not too sure how to resolve that, maybe this picture isn't perfect, or maybe the $b^?$ should be constrained to remove a d.o.f.?
One possible constraint would be to make them components of a unit vector.

\subsection*{Units for simulation}
In the simulation we are free to choose the units we work with, so that leaves 3 quantities we can choose as we like.
\par\noindent\fbox{\parbox{\textwidth}{
Inspecting Jack's code I'm pretty sure he used $C$, $b_1$ (in 1c.a.) and $\mu$, in the last version of his code I suspect he set $C$ to 2 (maybe to test something) and $\mu$ and $b_1$ to 1.
}}

Okay, so I'm kinda struggling to find the best way to do this, so let me just define a simple way to do it for the computations.
Use units such that in them $C, b_1$ and $\mu$ each have the value 1 in their respective units, then
\par\noindent\begin{minipage}{.5\linewidth}
\begin{align}
    C =& \frac{E}{L^3} \nonumber \\
    b_1 =& \frac{E}{L} \\
    \mu =& \frac{1}{E T} \nonumber
\end{align}
\end{minipage}
\noindent\begin{minipage}{.5\linewidth}
\begin{align}
     L &= \sqrt{\frac{b_1}{C}} \nonumber \\
     E &= b_1 L = \sqrt{\frac{b_1^3}{C}} \\
     T &= \frac{1}{\mu E} = \frac{1}{\mu b_1 L} = \frac{1}{\mu} \sqrt{\frac{C}{b_1^3}} \nonumber
\end{align}
\end{minipage}

\subsubsection*{Resulting physical quantities}
I also list the formulas for the quantities Jack discussed in section 3.5 of his thesis, here I'm a little unsure about the correctness however (say, I'm not sure if I should propagate the changes to the $\frac{1}{2}$ factors or not), one thing is that I'm not sure how the $\parallel$ and $\perp$ parts should be involved so what I quote is only for the 1c.a. case.
Here the lowercase letters $a$ and $b$ represent the numeric values used in the code for $A$ and $b_2$.
\begin{align}
    \varepsilon &= \sqrt{\frac{b_1}{|A|}} = \sqrt{\frac{1}{|a|}} L \\
    \lambda &= \sqrt{\frac{b_2}{b_1}} = \sqrt{b} L \\
    \kappa &= \frac{\lambda}{\varepsilon} = \sqrt{\frac{b}{|a|}}
\end{align}

\subsubsection*{Implementation}
This leaves 2 dimensionfull parameters, $A$ which can be positive or negative and $b_2$ which can only be positive.
$A$ is specified directly as $a$ which, as in the previous bit, is the numerical value in the units above.
$b_2$ is specified through the square of the Ginzburg parameter $\kappa^2$ (dimensionless, see above).

Later on, when the projection operators are implemented, the two dimensionless $b^?$ will also be added.


\vspace{3em}
\subsection*{Bit of a discussion that I'm not sure goes anywhere}
Now for a uniaxial smectic we have $|\du{E}|^2 = \frac{d-1}{d}||\psi||^2$
Clearly for a fully isotropic phase $\du{E} = 0 \rightarrow \psi = 0$, if we want to "define" $\psi$ such that it is 1 in a fully smectic phase.

\end{document}
