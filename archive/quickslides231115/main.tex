\documentclass[10pt]{beamer}

\usepackage[utf8]{inputenc}
\usepackage{parskip}
\usepackage{amsmath}
\usepackage{physics}
\usepackage{stackengine}
\usepackage{float}
\usepackage{cleveref}
\usepackage{siunitx}

\allowdisplaybreaks

\newcommand\textbff[1]{\textbf{\boldmath #1}}
\newcommand\circled[1]{\raisebox{.5pt}{\textcircled{\raisebox{-.9pt} {#1}}}
    }

% We often use underlines here
\newcommand\barbelow[1]{\stackunder[1.2pt]{$#1$}{\rule{1.5ex}{.1ex}}}
\newcommand{\su}[1]{\barbelow{#1}}
\newcommand{\du}[1]{\barbelow{\barbelow{#1}}}

\newcommand{\pp}{\partial}

\newcommand{\YY}[3][j]{E_{#2#1}E_{#3#1}^*}

\def\onedot{$\mathsurround0pt\ldotp$}
\def\cddot{\mathbin{
    \vcenter{\baselineskip1ex \vspace{-0.1ex}\hbox{\onedot}\hbox{\onedot}}
}}


\begin{document}

\begin{frame}
\frametitle{E theory}
\begin{itemize}
    \item Building a smectic analogoue to nematic Q tensor
    \item Replace the real order parameters $S$ with complex $\psi$
    \item Mainly thinking about uniaxial case
\end{itemize}
\begin{align*}
    \du{Q} &= S_1(\su{N}\su{N} - \frac{\du{\delta}}{d}) + \underset{\text{$S_2=0$ makes it uniaxial}}{\boxed{S_2(\su{M}\su{M} - \frac{\du{\delta}}{d})}} \\
    \du{E} &\sim \psi_1(\su{N}\su{N} - \frac{\du{\delta}}{d})
\end{align*}
\begin{itemize}
    \item This allows the order to numerically melt at defects
    \item Preserves the $\su{N}\rightarrow-\su{N}$ symmetry
\end{itemize}
\end{frame}

\begin{frame}
\frametitle{Relation to the density fluctuations}
\begin{itemize}
    \item $|\psi|$ represents quality of layers
    \item $q_0$ the base layering and $\phi$ the local offset from it
    \item $\su{N}$ is the director, normal to the layering
\end{itemize}
\begin{align*}
    \rho \sim \rho_0 + 2\Re(|\psi|e^{i\phi} e^{i q_0 \su{N} \cdot \su{r}})
\end{align*}
\begin{itemize}
    \item In \EE\ theory these are all degrees of freedom, unlike when $\su{N} = \frac{\su{\nabla}\phi}{|\su{\nabla}\phi|}$ is used
    \item Using \EE\ makes melting in defect cores easier in numerical simulations
\end{itemize}
\end{frame}

\begin{frame}
\frametitle{Thinking about constraints, and biaxiality}
\begin{itemize}
    \item Q is real, so symmetry makes it diagonalizable -- this leads to the biaxial form
    \item If a complex matrix is normal ($\du{E}\du{E}^\dagger=\du{E}^\dagger\du{E}$) it can be diagonalized by a unitary matrix
\end{itemize}
\begin{align*}
    \du{E} &= \du{U}^\dagger \begin{pmatrix}
        \lambda_1 & 0 & 0 \\
        0 & \lambda_2 & 0 \\
        0 & 0 & - \lambda_1 - \lambda_2
    \end{pmatrix} \underset{\text{also using symmetry}}{\du{U} = \cdots = \psi_1(\su{N}}\su{N} - \frac{\du{\delta}}{d}) + \psi_2(\su{M}\su{M} - \frac{\du{\delta}}{d})
\end{align*}
\begin{itemize}
    \item With $\su{N}$, $\su{M}$ being real, orthogonal, unit vectors, and
\end{itemize}
\begin{align*}
    \begin{pmatrix} \psi_1 \\ \psi_2 \end{pmatrix} = \begin{pmatrix} 2 & 1 \\ 1 & 2 \end{pmatrix} \begin{pmatrix} \lambda_1 \\ \lambda_2 \end{pmatrix} = \begin{pmatrix} \lambda_1 - \lambda_3 \\ \lambda_2 - \lambda_3 \end{pmatrix}
\end{align*}
\begin{itemize}
    \item Overall 2*2 + 2 + 1 = 7 dof, 4 if uniaxial
\end{itemize}
\end{frame}

\setlength{\abovedisplayskip}{1em}
\setlength{\belowdisplayskip}{0ex}

% \begin{frame}
% \frametitle{The free energy}
% \begin{itemize}
%     \item Using the simplest terms
%     \item We need to match $\du{E}$ and $\du{E}^*^$ to make it real
%     \item Take them to be functions of $|?_{ij...k}|^2=?_{ij...k}?_{ij...k}^*$
%     \item Only use $\du{E}$ (and $\su{\nabla}$)
%     \item As few $\du{E}$ as possible
% \end{itemize}
% \begin{align*}
%     f_\text{bulk}(E_{ij}E_{ij}^* = \Tr(\du{E}\du{E}^*)) = \frac{A}{2} E_{ij}E_{ij}^* + \frac{C}{4} (E_{ij}E_{ij}^*)^2
% \end{align*}
% \begin{itemize}
%     \item Elastic terms, need gradients -- below are "single elastic constant" terms
% \end{itemize}
% \begin{align*}
%     &|\su{\nabla}\du{E}|^2 = E_{ij,k}E_{ij,k}^* \qq{is the one we consider, nice to work with} \\
%     &|\su{\nabla}\cdot\du{E}|^2 = E_{ij,j}E_{ik,k}^* \qq{might be worth looking into, is not as nice} \\
%     &|\nabla^2\du{E}|^2 = E_{ij,kk}E_{ij,ll}^* \qq{the only double gradient term considered}
% \end{align*}
% \end{frame}
%
\begin{frame}
\frametitle{The free energy}
\begin{itemize}
    \item Using the simplest terms
    \item We need to match $\du{E}$ and $\du{E}^*^$ to make it real
\end{itemize}
\begin{align*}
    f_\text{bulk}(E_{ij}E_{ij}^* = \Tr(\du{E}\du{E}^*)) = \frac{A}{2} E_{ij}E_{ij}^* + \frac{C}{4} (E_{ij}E_{ij}^*)^2
\end{align*}
\begin{itemize}
    \item Elastic terms need gradients -- below are "single elastic constant" terms
\end{itemize}
\begin{align*}
    &|\su{\nabla}\du{E}|^2 = E_{ij,k}E_{ij,k}^* \qq{the main term} \\
    &|\su{\nabla}\cdot\du{E}|^2 = E_{ij,j}E_{ik,k}^* \qq{not considered in Jacks' work, possibly surface term?} \\
    &|\nabla^2\du{E}|^2 = E_{ij,kk}E_{ij,ll}^* \qq{the only double gradient term considered}
\end{align*}
\end{frame}

\begin{frame}
\frametitle{Projection operators}
\begin{itemize}
    \item Gradients in different directions have different energy costs
    \item Working with uniaxial \EE\ (or nearly so) -- special direction is $\su{N}$
    \item Projection operator $\du{\Pi}=\su{N}\su{N}$
    \item Perpendicular projections are then $\du{T} = \du{\delta} - \du{\Pi}$
    \item Adapt the free energies by $\su{\nabla} \rightarrow \du{\Pi}\cdot\su{\nabla} + \du{T}\cdot\su{\nabla}$
\end{itemize}
% \vspace{1em}
\begin{align*}
    &E_{ij,k}E_{ij,k}^* \rightarrow f_\text{comp} = b_1^\parallel \Pi_{kl} E_{ij,k}E_{ij,l}^* + b_1^\perp T_{kl} E_{ij,k}E_{ij,l}^* \\
    &E_{ij,kk}E_{ij,ll}^* \rightarrow f_\text{curv} = b_2^\parallel \Pi_{kl}E_{ij,lk}\Pi_{mn}E_{ij,nm}^* + b_2^\perp T_{kl}E_{ij,lk}T_{mn}E_{ij,nm}^* \\
    &\phantom{E_{ij,kk}E_{ij,ll}^* \rightarrow f_\text{curv} =}+ b_2^{\parallel\perp}(\Pi_{kl}E_{ij,lk}T_{mn}E_{ij,nm}^* + T_{kl}E_{ij,lk}\Pi_{mn}E_{ij,nm}^*)
\end{align*}
\end{frame}

\begin{frame}
\frametitle{Projection operators}
\begin{itemize}
    \item Need a form for \PP\ in terms of \EE
    \item Have 2 forms which work for uniaxial \EE
\end{itemize}
\begin{align*}
    \du{\Pi} =& \sqrt{\frac{d-1}{d \du{E} \cddot \du{E}}} \du{E} + \frac{\du{\delta}}{d} \\
    \du{\Pi} =& \frac{d-1}{d-2}\qty(\frac{\du{E} \cdot \du{E}^*}{\du{E} \cddot \du{E}^*} - \frac{\du{\delta}}{d(d-1)})
\end{align*}
\begin{itemize}
    \item Lead to seemingly different functional derivatives -- why?
    \item First form only has $\du{E}$, how about $\du{E} \rightarrow \du{E}^*$?
    \item How well do they work for biaxial \EE?
\end{itemize}
\end{frame}

\begin{frame}
\frametitle{Dynamics and functional derivatives}
\begin{itemize}
    \item Starting from $\mu \pdv{E_{ij}}{t} = -\fdv{F}{E_{ij}^*}$, but need to preserve constraints
    \item If $\fdv{F}{E_{ij}^*}$ is symmetric and traceless, then so will $\du{E}$
    \item Either treat $\du{E}$ as symmetric, or symmetrize after
% \end{itemize}
% \begin{align*}
% \end{align*}
% \begin{itemize}
    \item Normality is more complicated, Lagrange multiplier from Djorde's work
    \item It might be nice to constrain it to be unaxial too
\end{itemize}
\end{frame}

\begin{frame}
\frametitle{Functional derivatives}
\begin{itemize}
    \item Results using the square root version of \PP
\end{itemize}
\begin{align*}
    \fdv{F_\text{bulk}}{E_{ij}^*} =&\enspace \frac{1}{2}\qty(A + C E_{ab}E_{ab}^*)E_{ij} \\
    \fdv{F_\text{comp}}{E_{ij}^*} =&\enspace -(b_1^\parallel - b_1^\perp) (\Pi_{kl,l} E_{ij,k} + \Pi_{kl} E_{ij,kl}) - b_1^\perp E_{ij,kk} \\
    \fdv{F_\text{curv}}{E_{ij}^*} =&\enspace (b_2^\parallel + b_2^\perp - 2b_2^{\parallel\perp}) \Bigl( (\Pi_{kl}\Pi_{po,po} + 2\Pi_{kl,o}\Pi_{po,p} + \Pi_{kl,po}\Pi_{po})E_{ij,lk} \\
    &\phantom{\enspace (b_2^\parallel + b_2^\perp - 2b_2^{\parallel\perp}) \Bigl(}+ 2(\Pi_{kl,o}\Pi_{po} + \Pi_{kl}\Pi_{po,o})E_{ij,lkp} + \Pi_{kl}\Pi_{po}E_{ij,lkpo} \Bigr) \nonumber \\
    &+ (b_2^{\parallel\perp} - b_2^\perp)\Bigl( \Pi_{po,po}E_{ij,kk} + 2\Pi_{po,o}E_{ij,kkp} + \Pi_{po}E_{ij,kkpo} \nonumber \\ 
    &\phantom{\enspace+ (b_2^{\parallel\perp} - b_2^\perp)\Bigl(}+ \Pi_{kl,oo}E_{ij,lk} + 2\Pi_{kl,o}E_{ij,lko} + \Pi_{kl}E_{ij,lkoo} \Bigr) \nonumber \\ 
    &+ b_2^\perp E_{ij,kkoo} \nonumber
\end{align*}
\end{frame}

\begin{frame}
\frametitle{Functional derivatives}
\begin{itemize}
    \item Results using the square root version of \PP
\end{itemize}
\begin{align*}
    \Pi_{kl} =&\enspace \frac{s E_{kl}}{\sqrt{E_{ab}E_{ab}}} + \frac{\delta_{kl}}{d} \\
    \Pi_{kl,m} =&\enspace \frac{s}{\sqrt{E_{ab}E_{ab}}} \qty(E_{kl,m} - \frac{E_{kl}E_{cd}E_{cd,m}}{E_{ab}E_{ab}}) \\
    \Pi_{kl,mn} =&\enspace \frac{s}{\sqrt{E_{ab}E_{ab}}} \Biggl(E_{kl,mn} \\ 
    &-\frac{E_{kl,n}E_{cd}E_{cd,m} + E_{kl,m}E_{cd}E_{cd,n} + E_{kl}(E_{cd,n}E_{cd,m} + E_{cd}E_{cd,mn})}{E_{ab}E_{ab}} \\
    &\phantom{\frac{s}{\sqrt{E_{ab}E_{ab}}} \Biggl(} + 3\frac{E_{kl}E_{cd}E_{cd,m}E_{ef}E_{ef,n}}{(E_{ab}E_{ab})^2} \Biggr)
\end{align*}
\end{frame}

\end{document}
