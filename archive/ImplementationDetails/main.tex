\documentclass[11pt]{article}
\usepackage[top=30mm,bottom=30mm,left=20mm,right=20mm]{geometry}
\usepackage[utf8]{inputenc}
\usepackage{parskip}
\usepackage{amsmath}
\usepackage{physics}
\usepackage{stackengine}
\usepackage{float}
\usepackage{cleveref}
\usepackage{graphics}
\usepackage{siunitx}

\allowdisplaybreaks

\newcommand\textbff[1]{\textbf{\boldmath #1}}
\newcommand\circled[1]{\raisebox{.5pt}{\textcircled{\raisebox{-.9pt} {#1}}}
    }
\newcommand{\shortnote}[1]{\textit{\footnotesize (#1)}}

\newcommand{\pp}{\partial}
\newcommand{\mgrad}{\su{\nabla}}

% We often use underlines here
\newcommand\barbelow[1]{\stackunder[1.2pt]{$#1$}{\rule{1.5ex}{.1ex}}}
\newcommand{\su}[1]{\barbelow{#1}}
\newcommand{\du}[1]{\barbelow{\barbelow{#1}}}

\newcommand{\EE}{$\du{E}$}
\newcommand{\PP}{$\du{\Pi}$}
\newcommand{\ddelta}[4]{\delta_{#1#3}\delta_{#2#4} + \delta_{#1#4}\delta_{#2#3}}

\newcommand{\YY}[3][j]{E_{#2#1}E_{#3#1}^*}

\def\onedot{$\mathsurround0pt\ldotp$}
\def\cddot{\mathbin{
    \vcenter{\baselineskip1ex \vspace{-0.1ex}\hbox{\onedot}\hbox{\onedot}}
}}


\interdisplaylinepenalty=3000

\begin{document}
\begin{center}
    \LARGE
    \textbf{Implementation details and non-dimensionalization}
\end{center}
\vspace{1em}
\section{Full version with projection operators}
This is the main version of the code and structs are designed for this, below is also a description of the one constant approximation version which uses some of the fields from below for its variables.
Here I use:
\begin{align}
    f_\text{bulk} =&\enspace A E_{ij}E_{ij}^* + \frac{C}{2} (E_{ij}E_{ij}^*)^2 \\
    f_\text{comp} =&\enspace b_1^\parallel \Pi_{kl} E_{ij,k}E_{ij,l}^* + b_1^\perp T_{kl} E_{ij,k}E_{ij,l}^* \\
    f_\text{cdiv} =&\enspace b_d |\su{\nabla} \cdot \du{E}|^2 = b_d E_{ji,j}E_{ji,j}^* \qq{No \PP\ for now}\\
    f_\text{curv} =&\enspace b_2^\parallel \Pi_{kl}E_{ij,lk}\Pi_{mn}E_{ij,nm}^* + b_2^\perp T_{kl}E_{ij,lk}T_{mn}E_{ij,nm}^* + b_2^{\parallel\perp}(\Pi_{kl}E_{ij,lk}T_{mn}E_{ij,nm}^* + T_{kl}E_{ij,lk}\Pi_{mn}E_{ij,nm}^*)
\end{align}
where $C$, and all the $b$s are positive, but $A$ can be negative.
And a time evolution of form
\begin{equation}
    \pdv{\du{E}}{t} = -\mu \fdv{F}{\du{E}^*}
\end{equation}
\noindent\fbox{\parbox{\textwidth}{
    Notable differences from Jack's are that I omit the 2 extra factors of $\frac{1}{2}$ in the bulk contribution, change $\mu$ to its inverse and add the divergence term (it can be set to 0).
}}

Three dimensions come up -- energy ($E$), length ($L$) and time ($T$) and the quantities above have units as follows:
\begin{center}
    \renewcommand{\arraystretch}{1.5}
    \begin{tabular}{c | c | c | c | c | c }
        quantity & \EE & $A$, $C$ & $b_1^?$, $b_d$ & $b_2^?$ & $\mu$ \\
        \hline
        unit & 1 & $\frac{E}{L^3}$ & $\frac{E}{L}$ & $E L$ & $\frac{1}{E T}$ \\
    \end{tabular}
\end{center} 

\subsection{Physical quantities}
These are taken straight from Jack's, I do not account for the change of a $\frac{1}{2}$ factor in $A$ and $C$ as they are order of magnitude numbers anyway.
\begin{align}
    |\psi|_{eq} &= \sqrt{\frac{3}{2} * \frac{-A}{C}} \qq{The ideal smectic phase value}\\
    \varepsilon &= \sqrt{\frac{b_1^\parallel}{|A|}} \qq{Lamellar in-plane coherence length} \\
    \lambda &= \sqrt{\frac{b_2^\perp}{b_1^\parallel}} \qq{Penetration depth} \\
    \kappa &= \frac{\lambda}{\varepsilon} = \sqrt{\frac{b_2^\perp|A|}{b_1^\parallel^2}} \qq{Ginzburg parameter}
\end{align}

\subsection{Non-dimensionalization}
In the end I decided the simulation itself is best ran with all the constants above (they are all stored in the \verb!lcParam! struct) so that things are easy to compare and the non-dimensionalization choices can be changed relatively easily.

The non-dimensionalization is however still implemented, just before the simulation itself.
\textbf{Currently, the requirements are that $|\psi|_{eq} = 1$, $\varepsilon = 1$ and only allow $A$ to be $\pm 1$ (or 0).}
For the $A \neq 0$ cases this implies $C = \frac{3}{2}$ and $b_1^\parallel = 1$.
Out of the remaining parameters we can set $\mu = 1$ which will specify the time units and the rest needs to be set.
This way $A, b_1^\parallel$ and $\mu$ are what sets the units as follows:
\begin{align}
    L = \sqrt{\frac{b_1^\parallel}{|A|}}, \quad\quad E = b_1 L = \sqrt{\frac{b_1^\parallel^3}{|A|}}, \quad\quad T = \frac{1}{\mu E} = \frac{1}{\mu} \sqrt{\frac{|A|}{b_1^\parallel^3}}
\end{align}

I haven't actually figured out the $A=0$ case currently.

\vspace{1em}
\Large \textbf{UPDATE} \normalsize

So I changed the above now so that I can better explore different params.
I now allow $A$ to be set to any value and set $C$ relatively to it to make the bulk energy minimum at $|\psi_1| = 1$ (if negative $A$).
I allow both $b_1^?$ values to be set to any non-negative numbers directly.
For the $b_2^?$ values I still have a Ginzburg parameter input, but then I also have one input for each $b_2^?$ and they are scaled by $\sqrt{K}$.

\subsection{Implementation}
So $A, C, b_1^\parallel$ and $\mu$ are set already from units, then we can still use the Ginzburg parameter to set $b_2^\perp$ and set $b_d$ directly as it is an extra for now.
Finally then I set the other $b_1$ value via $b_1^\parallel$ and respectively with $b_2^\perp$.


\section{One constant approximation version}
Here I use the simplified free energies:
\begin{align}
    f_\text{bulk} =&\enspace A |\du{E}|^2 + \frac{C}{2} |\du{E}|^4 \\
    f_\text{comp} =&\enspace b_1 |\su{\nabla}\du{E}|^2 \\
    f_\text{cdiv} =&\enspace b_d |\su{\nabla} \cdot \du{E}|^2 \\
    f_\text{curv} =&\enspace b_2 |\nabla^2\du{E}|^2 \\
\end{align}
where $C$, and all the $b$s are positive, but $A$ can be negative.
And a time evolution of form
\begin{equation}
    \pdv{\du{E}}{t} = -\mu \fdv{F}{\du{E}^*}
\end{equation}
\noindent\fbox{\parbox{\textwidth}{
    Still holds that notable differences from Jack's are that I omit the 2 extra factors of $\frac{1}{2}$ in the bulk contribution, change $\mu$ to its inverse and add the divergence term (it can be set to 0).
}}
\begin{center}
    \renewcommand{\arraystretch}{1.5}
    \begin{tabular}{c | c | c | c | c | c | c | c }
        quantity & \EE & $A$ & $C$ & $b_1$ & $b_d$ & $b_2$ & $\mu$ \\
        \hline
        unit & 1 & $\frac{E}{L^3}$ & $\frac{E}{L^3}$ & $\frac{E}{L}$ & $\frac{E}{L}$ & $E L$ & $\frac{1}{E T}$ \\
    \end{tabular}
\end{center} 

\subsection{Physical quantities}
Here I adopt the quantities from above as
\begin{align}
    |\psi|_{eq} &= \sqrt{\frac{3}{2} * \frac{-A}{C}} \qq{The ideal smectic phase value}\\
    \varepsilon &= \sqrt{\frac{b_1}{|A|}} \qq{Lamellar in-plane coherence length} \\
    \lambda &= \sqrt{\frac{b_2}{b_1}} \qq{Penetration depth} \\
    \kappa &= \frac{\lambda}{\varepsilon} = \sqrt{\frac{b_2|A|}{b_1^2}} \qq{Ginzburg parameter}
\end{align}

\subsection{Units and non-dimensionalization for simulation}
\textbf{Exactly as above, the requirements are that $|\psi|_{eq} = 1$, $\varepsilon = 1$ and only allow $A$ to be $\pm 1$ or 0.}
For the $A \neq 0$ cases this implies $C = \frac{3}{2}$ and $b_1 = 1$.
Out of the remaining parameters we can set $\mu = 1$ which will specify the time units, set $b_2$ via the Ginzburg parameter and $b_d$ directly.
This way $A, b_1$ and $\mu$ are what sets the units as follows:
\begin{align}
    L = \sqrt{\frac{b_1}{|A|}}, \quad\quad E = b_1 L = \sqrt{\frac{b_1^3}{|A|}}, \quad\quad T = \frac{1}{\mu E} = \frac{1}{\mu} \sqrt{\frac{|A|}{b_1^3}}
\end{align}
I haven't actually figured out the $A=0$ case as of now.

\subsection{Summary}
So $A, b_1$ and $\mu$ are used to set the units, $|\psi|$ is in the 0 to 1 range and coherence length is 1$L$ which sets $C$ to $\frac{3}{2}$.
Then the user specifies $K$ to set $b_2$ and possibly a non-zero $b_d$.


\end{document}
