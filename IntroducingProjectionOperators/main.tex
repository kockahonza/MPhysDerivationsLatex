\documentclass[11pt]{article}
\usepackage[top=40mm,bottom=40mm,left=20mm,right=20mm]{geometry}

\usepackage[utf8]{inputenc}
\usepackage{parskip}
\usepackage{amsmath}
\usepackage{physics}
\usepackage{stackengine}
\usepackage{float}
\usepackage{cleveref}
\usepackage{graphics}
\usepackage{siunitx}

\allowdisplaybreaks

\newcommand\textbff[1]{\textbf{\boldmath #1}}
\newcommand\circled[1]{\raisebox{.5pt}{\textcircled{\raisebox{-.9pt} {#1}}}
    }
\newcommand{\shortnote}[1]{\textit{\footnotesize (#1)}}

\newcommand{\pp}{\partial}
\newcommand{\mgrad}{\su{\nabla}}

% We often use underlines here
\newcommand\barbelow[1]{\stackunder[1.2pt]{$#1$}{\rule{1.5ex}{.1ex}}}
\newcommand{\su}[1]{\barbelow{#1}}
\newcommand{\du}[1]{\barbelow{\barbelow{#1}}}

\newcommand{\EE}{$\du{E}$}
\newcommand{\PP}{$\du{\Pi}$}
\newcommand{\ddelta}[4]{\delta_{#1#3}\delta_{#2#4} + \delta_{#1#4}\delta_{#2#3}}

\newcommand{\YY}[3][j]{E_{#2#1}E_{#3#1}^*}

\def\onedot{$\mathsurround0pt\ldotp$}
\def\cddot{\mathbin{
    \vcenter{\baselineskip1ex \vspace{-0.1ex}\hbox{\onedot}\hbox{\onedot}}
}}


\begin{document}
\begin{center}
    \LARGE
    \textbf{A way of justifying the free energy form, and introducing projection operators}
\end{center}
\vspace{1em}

\section{Free energy by itself, using the Frobenius norm}
The starting point here is to assume all contributions to the free energy will only involve \EE\ and \mgrad, and that each term will of a Frobenius normal form (possibly generalized to higher rank tensors, I'm not sure).
The second part is to make sure the free energy is real.
(This might not be entirely true, there could be Levi-civita symbols, or others, involved too, like in the $\du{Q}$ tensor, but it is a starting point)

Then we can classify the terms in order of how many \mgrad\ appear, use \emph{bulk} for 0 occurrences, \emph{comp} for 1 and \emph{curv} for 2.
I don't consider more than 2 occurrences here.
Some of the resulting possible terms are then summarized below, in order of number of required \EE\ occurrences.

\paragraph{Bulk terms} give only one possibility with a single \EE\ and that is $|\du{E}|^2$ (as $\Tr(\du{E}) = 0$), and similar to the $\du{Q}$ tensor we then consider the bulk free energy to be a scalar function of this term.
However, if we were to go further we could have $|\du{E}\du{E}|^2 = |\du{E}|^4$ so that is actually already accounted for. Then we have $|\du{E} \cdot \du{E}|^2$ and even $|\du{E} \cddot \du{E}|^2$ neither of which I think is in general equivalent to the previous, I also tried the UA case and nothing clear came out.
However maybe they can be accounte for using the scalar function using multiple terms (akin to a Taylor expansion)?

\paragraph{Comp terms} give two terms $|\su{\nabla} \du{E}|^2$ and $|\su{\nabla} \cdot \du{E}|^2$ for a single \EE\ and I don't go further.
Currently we only use the first, but as there is only one extra term it might be worth adding it.

\paragraph{Curv terms} give $|\su{\nabla}\su{\nabla}\du{E}|^2$ with a rank 4 tensor inside, $|\nabla^2 \du{E}|^2$ and $|\su{\nabla}(\su{\nabla} \cdot \du{E})|^2$ having a rank 2 tensor, and $|\su{\nabla} \cdot (\su{\nabla} \cdot \du{E})|^2$ having a scalar inside (these are the only unique possibilities when accounting for a symmetric, traceless \EE).
Now here there are quite a lot of options, and we only use the second.

\section{Introducing projection operators}
Now coming from the idea that different directions of \mgrad\ should be treated differently, and that we specifically want to account for one direction $\su{N}$ to be special.
We can represent it using $\du{\Pi} = \su{N}\su{N}$ and define $\du{T} = \du{\delta} - \du{\pi}$ to account for the rest.
Then we want to incorporate the projection operators into the energy terms involving the gradient, consider doing $\su{\nabla} \rightarrow b^\parallel \du{\Pi} \cdot \su{\nabla} + b^\perp \du{T} \cdot \su{\nabla}$.

Then using the properties that $\du{\Pi} \cdot \du{\Pi} = \du{\Pi}$, $\du{\Pi} \cdot \du{T} = 0$ and $\du{T} \cdot \du{T} = \du{T}$ we get the following free energy terms
\begin{align}
    |\su{\nabla}\du{E}|^2 &\rightarrow b^\parallel^2 \Pi_{kl} E_{ij,k}E_{ij,l} + b^\perp^2 T_{kl} E_{ij,k}E_{ij,l} \\
    |\nabla^2\du{E}|^2 &\rightarrow b^\parallel^4 \Pi_{kl}E_{ij,lk}\Pi_{mn}E_{ij,nm}^* + b^\perp^4 T_{kl}E_{ij,lk}T_{mn}E_{ij,nm}^* + b^\parallel^2 b^\perp^2 (\Pi_{kl}E_{ij,lk}T_{mn}E_{ij,nm}^* + T_{kl}E_{ij,lk}\Pi_{mn}E_{ij,nm}^*)
\end{align}


\end{document}
