\documentclass[11pt]{article}
\usepackage[top=40mm,bottom=40mm,left=20mm,right=20mm]{geometry}
\usepackage[utf8]{inputenc}
\usepackage{parskip}
\usepackage{amsmath}
\usepackage{physics}
\usepackage{stackengine}
\usepackage{float}
\usepackage{cleveref}
\usepackage{graphics}
\usepackage{siunitx}

\allowdisplaybreaks

\newcommand\textbff[1]{\textbf{\boldmath #1}}
\newcommand\circled[1]{\raisebox{.5pt}{\textcircled{\raisebox{-.9pt} {#1}}}
    }
\newcommand{\shortnote}[1]{\textit{\footnotesize (#1)}}

\newcommand{\pp}{\partial}
\newcommand{\mgrad}{\su{\nabla}}

% We often use underlines here
\newcommand\barbelow[1]{\stackunder[1.2pt]{$#1$}{\rule{1.5ex}{.1ex}}}
\newcommand{\su}[1]{\barbelow{#1}}
\newcommand{\du}[1]{\barbelow{\barbelow{#1}}}

\newcommand{\EE}{$\du{E}$}
\newcommand{\PP}{$\du{\Pi}$}
\newcommand{\ddelta}[4]{\delta_{#1#3}\delta_{#2#4} + \delta_{#1#4}\delta_{#2#3}}

\newcommand{\YY}[3][j]{E_{#2#1}E_{#3#1}^*}

\def\onedot{$\mathsurround0pt\ldotp$}
\def\cddot{\mathbin{
    \vcenter{\baselineskip1ex \vspace{-0.1ex}\hbox{\onedot}\hbox{\onedot}}
}}


\begin{document}
\begin{center}
    \LARGE
    \textbf{Derivation of symmetrized $\fdv{F}{E_{ij}^*}$ in terms of $E_{ij}$ and its derivatives}
\end{center}
\vspace{1em}
\section{Initial setup}
\subsection{Functional derivatives and independent variables}
Here I assume all the components of \EE\ are independent, except symmetry, and that \EE\ and $\du{E}^*$ are independent as usual with Wirtinger derivatives.
The reason for this is essentially as symmetry is easy to include, unlike the others which will be enforced for the dynamics using Lagrange multipliers, at least for now.

To be entirely clear these assumptions can be summarized as $\pdv{E_{ij}}{E_{ab}} = \kappa(\delta_{ai}\delta_{bj} + \delta_{aj}\delta_{bi})$ and $\pdv{E_{ij,k}}{E_{ab,c}} = \kappa(\delta_{ai}\delta_{bj} + \delta_{aj}\delta_{bi})\delta_{kc}$ (with $\kappa$ being there as I'm currently unsure about a factor of 2).
Let me also note that I strongly suspect this will yield exactly twice the result as any $\fdv{F}{E_{ij}^*}$ using $\pdv{E_{ij}}{E_{ab}}=\delta_{ai}\delta_{bj}$ etc. that happens to be symmetric.

My doubts as to do with a possible $\kappa$ is that when $i\neq j$ we expect a contribution of 1 for $a=i\qq{and}b=j$ and another contribution of 1 for $a$ and $b$ switched, however if $i=j$ there should only be 1 contribution of 1, this is visualized by \cref{tab:devtable}.
For now however just proceed and hope it isn't a problem.
\begin{table}[H]
    \begin{center}
        \begin{tabular}{ c|c c c c c c c c c } 
               & 11 & 22 & 33 & 12 & 21 & 13 & 31 & 23 & 32 \\
            \hline
            11 & 1  &  0 &  0 &  0 &  0 &  0 &  0 &  0 &  0 \\
            22 & 0  &  1 &  0 &  0 &  0 &  0 &  0 &  0 &  0 \\
            33 & 0  &  0 &  1 &  0 &  0 &  0 &  0 &  0 &  0 \\
            12 & 0  &  0 &  0 &  1 &  1 &  0 &  0 &  0 &  0 \\
            21 & 0  &  0 &  0 &  1 &  1 &  0 &  0 &  0 &  0 \\
            13 & 0  &  0 &  0 &  0 &  0 &  1 &  1 &  0 &  0 \\
            31 & 0  &  0 &  0 &  0 &  0 &  1 &  1 &  0 &  0 \\
            23 & 0  &  0 &  0 &  0 &  0 &  0 &  0 &  1 &  1 \\
            32 & 0  &  0 &  0 &  0 &  0 &  0 &  0 &  1 &  1
        \end{tabular}
    \end{center}
    \caption{The results of $\pdv{E_{ij}}{E_{ab}}$, rows and columns correspond to the $ij$ and $ab$ tuples and the table values to the partial derivative, clearly there is a difference between the $i=j$ and $i\neq j$ terms.}\label{tab:devtable}
\end{table}

\subsection{Free energies and projection operators}
\begin{align}
    F & = \int f_\text{bulk} + f_\text{comp} + f_\text{curv} \dd{V} = F_\text{bulk} + F_\text{comp} + F_\text{curv}\\
    f_\text{bulk} & = \frac{A}{2} E_{ij}E_{ij}^* + \frac{C}{4} (E_{ij}E_{ij}^*)^2 \\
    f_\text{comp} & = b_1^\parallel \Pi_{kl} E_{ij,k}E_{ij,l}^* + b_1^\perp T_{kl} E_{ij,k}E_{ij,l}^* \qq{maybe try adding} b_1^d E_{ij,j}E_{ik,k}^* \qq{later too} \\
    f_\text{curv} & = \quad \ldots \qq{for later} \ldots \\
\end{align}
where
\begin{align}
    \du{\Pi} = \su{N} \su{N} && \text{and} && \du{T} = \du{\delta} - \du{\Pi}
\end{align}
are the projection operators. We need to express these using $\du{E}$ as well, there are 2 options which I quote here
\begin{align}
    \du{\Pi} & = \frac{d-1}{d-2}\qty(\frac{\du{E} \cdot \du{E}^*}{\du{E} \cddot \du{E}^*} - \frac{\du{\delta}}{d(d-1)}) \qq{or} \label{eq:piex1}\\
    \du{\Pi} & = \sqrt{\frac{d-1}{d \du{E} \cddot \du{E}}} \du{E} - \frac{\du{\delta}}{d} \label{eq:piex2} \qq{which has a complex square root}
\end{align}
$\du{T}$ just being calculated from $\du{\Pi}$.

\section{$F_\text{bulk}$}
\subsection{Using $\kappa$}
\begin{align}
    \fdv{F_\text{bulk}}{E_{ij}^*} &= \frac{A\kappa}{2}(E_{ij} + E_{ji}) + \frac{B\kappa}{4} 2(E_{ab}E_{ab}^*)(E_{ij}+E_{ji}) \\
    &= \kappa(A + B E_{ab}E_{ab}^*) E_{ij} \qq{using symmetry of \EE}
\end{align}

\subsection{Treating the $i=j$ and $i\neq j$ separately, as according to \cref{tab:devtable}}
With \textbff{no sum on $i$}:
\begin{align}
    \fdv{F_\text{bulk}}{E_{ii}^*} &= \frac{A}{2}E_{ii} + \frac{B}{4} 2(E_{ab}E_{ab}^*)E_{ii} \\
    &= (A + B E_{ab}E_{ab}^*)\frac{E_{ii}}{2}
\end{align}
and clearly
\begin{align}
    \fdv{F_\text{bulk}}{E_{ij}^*} &= (A + B E_{ab}E_{ab}^*) E_{ij}
\end{align}
\section{$F_\text{comp}$ using \cref{eq:piex2}}
To simplify use $s=\sqrt{\frac{d-1}{d}}$ to get
\begin{equation}
    \Pi_{kl} = s \qty(E_{ab}E_{ab})^{-\frac{1}{2}}E_{kl} - \frac{\delta_{kl}}{d} \qq{and recall} f_\text{comp} = (b_1^\parallel - b_1^\perp) \Pi_{kl} E_{ij,k}E_{ij,l}^* + b_1^\perp E_{ij,k}E_{ij,k}^*
\end{equation}
So to get $\fdv{F_\text{comp}}{E_{ij}^*}$, the first term in the Euler-Lagrange equations will be 0 again as only gradients of \EE\ appear directly in the form of $f_\text{comp}$ and \PP\ only has non-conjugated elements of \EE\ appear.
Thus
\begin{align}
    \fdv{F_\text{comp}}{E_{ij}^*} &= -\pp_k \pdv{f_\text{comp}}{E_{ij,k}} = -\pp_k \qty((b_1^\parallel - b_1^\perp) \Pi_{cd} E_{ab,c} (\ddelta{i}{j}{a}{b})\delta_{kd} + b_1^\perp E_{ab,c}(\ddelta{i}{j}{a}{b})\delta_{kc}) \\
    &= -\pp_k \qty((b_1^\parallel - b_1^\perp) \Pi_{ck} (E_{ij,c} + E_{ji,c}) + (E_{ij,k} + b_1^\perp E_{ji,k})) \\
    &= -2\pp_k \qty((b_1^\parallel - b_1^\perp) \Pi_{ck} E_{ij,c} + b_1^\perp E_{ij,k}) \\
    &= -2\qty((b_1^\parallel - b_1^\perp) \Pi_{ck,k} E_{ij,c} + (b_1^\parallel - b_1^\perp) \Pi_{ck} E_{ij,ck} + b_1^\perp E_{ij,kk})
\end{align}
So we need
\begin{align}
    \Pi_{ck,k} &= \pp_k \qty( s \qty(E_{ab}E_{ab})^{-\frac{1}{2}}E_{ck} - \frac{\delta_{ck}}{d}) = s \pp_k \qty( \qty(E_{ab}E_{ab})^{-\frac{1}{2}}E_{ck}) \\
    &= s \qty(\qty(E_{ab}E_{ab})^{-\frac{1}{2}}E_{ck,k} - \frac{1}{2}\qty(E_{ab}E_{ab})^{-\frac{3}{2}}E_{ck}2E_{ab}E_{ab,k}) \\
    &= s \qty(\qty(E_{ab}E_{ab})^{-\frac{1}{2}}E_{ck,k} - \qty(E_{ab}E_{ab})^{-\frac{3}{2}}E_{ck}E_{ab}E_{ab,k}) \\
    &= \frac{s}{\sqrt{E_{ab}E_{ab}}} \qty(E_{ck,k} - \frac{E_{ab}E_{ab,k}}{E_{ab}E_{ab}}E_{ck})
\end{align}
so together we get
\begin{align}
    \fdv{F_\text{comp}}{E_{ij}^*} &= -2\qty(\frac{s(b_1^\parallel - b_1^\perp)}{\sqrt{E_{ab}E_{ab}}} \qty(E_{ck,k} - \frac{E_{ab}E_{ab,k}}{E_{ab}E_{ab}}E_{ck}) E_{ij,c} + (b_1^\parallel - b_1^\perp) E_{ij,ck} + b_1^\perp E_{ij,kk})
\end{align}

\end{document}
