\documentclass[11pt]{article}
\usepackage[top=40mm,bottom=40mm,left=20mm,right=20mm]{geometry}
\usepackage[utf8]{inputenc}
\usepackage{parskip}
\usepackage{amsmath}
\usepackage{physics}
\usepackage{stackengine}
\usepackage{float}
\usepackage{cleveref}
\usepackage{graphics}
\usepackage{siunitx}

\allowdisplaybreaks

\newcommand\textbff[1]{\textbf{\boldmath #1}}
\newcommand\circled[1]{\raisebox{.5pt}{\textcircled{\raisebox{-.9pt} {#1}}}
    }
\newcommand{\shortnote}[1]{\textit{\footnotesize (#1)}}

\newcommand{\pp}{\partial}
\newcommand{\mgrad}{\su{\nabla}}

% We often use underlines here
\newcommand\barbelow[1]{\stackunder[1.2pt]{$#1$}{\rule{1.5ex}{.1ex}}}
\newcommand{\su}[1]{\barbelow{#1}}
\newcommand{\du}[1]{\barbelow{\barbelow{#1}}}

\newcommand{\EE}{$\du{E}$}
\newcommand{\PP}{$\du{\Pi}$}
\newcommand{\ddelta}[4]{\delta_{#1#3}\delta_{#2#4} + \delta_{#1#4}\delta_{#2#3}}

\newcommand{\YY}[3][j]{E_{#2#1}E_{#3#1}^*}

\def\onedot{$\mathsurround0pt\ldotp$}
\def\cddot{\mathbin{
    \vcenter{\baselineskip1ex \vspace{-0.1ex}\hbox{\onedot}\hbox{\onedot}}
}}


\begin{document}
\begin{center}
    \LARGE
    \textbf{Derivation of symmetrized $\fdv{F}{E_{ij}^*}$ in terms of $E_{ij}$ and its derivatives}
\end{center}
\vspace{1em}
\section{Initial setup}
\subsection{Functional derivatives and independent variables}\label{sec:funcder}
Here I assume all the components of \EE\ are independent, except symmetry, and that \EE\ and $\du{E}^*$ are independent as usual with Wirtinger derivatives.
The reason for this is essentially as symmetry is easy to include, unlike the others which will be enforced for the dynamics using Lagrange multipliers, at least for now.

% Let me also note that I strongly suspect this will yield exactly twice the result as any $\fdv{F}{E_{ij}^*}$ using $\pdv{E_{ij}}{E_{ab}}=\delta_{ai}\delta_{bj}$ etc. that happens to be symmetric.
This can be achieved by a suitable choice of $\pdv{E_{ij}}{E_{ab}}$ (and respectively for gradients), in the previous derivation I used $\delta_{ia}\delta_{jb}$.
Here something of the form $\pdv{E_{ij}}{E_{ab}} = \kappa(\delta_{ai}\delta_{bj} + \delta_{aj}\delta_{bi})$ seems like a logical choice, with $\kappa$ being there due to the following problem.

This form however can't quite be made right for all $i,j,a,b$.
Consider $i=j$, then the only non-zero choice for $a,b$ is $a=b=i=j$, this then gets a contribution of $2\kappa$, but it should be 1 as clearly $\pdv{E_{ii}}{E_{ii}} = \pdv{x}{x}=1$ (no sums).
Thus to get the $i=j$ terms right we would need $\kappa=\frac{1}{2}$.

However then for any $i\neq j$ we would get two non-zero options, either $a=i,b=j$ or $a=j,b=i$, for each of these the formula above gives $\kappa$.
However for each of these we should get $\pdv{E_{ij}}{E_{ij}} = \pdv{E_{ij}}{E_{ij}} = 1$, this means $\kappa$ would have to be 1.

Essentially the diagonal and off-didagonal elements of \EE\ would have to be treated differently, all of this can be nicely summarized by \cref{tab:devtable}.
\begin{table}[H]
    \begin{center}
        \begin{tabular}{ c|c c c c c c c c c } 
               & 11 & 22 & 33 & 12 & 21 & 13 & 31 & 23 & 32 \\
            \hline
            11 & 1  &  0 &  0 &  0 &  0 &  0 &  0 &  0 &  0 \\
            22 & 0  &  1 &  0 &  0 &  0 &  0 &  0 &  0 &  0 \\
            33 & 0  &  0 &  1 &  0 &  0 &  0 &  0 &  0 &  0 \\
            12 & 0  &  0 &  0 &  1 &  1 &  0 &  0 &  0 &  0 \\
            21 & 0  &  0 &  0 &  1 &  1 &  0 &  0 &  0 &  0 \\
            13 & 0  &  0 &  0 &  0 &  0 &  1 &  1 &  0 &  0 \\
            31 & 0  &  0 &  0 &  0 &  0 &  1 &  1 &  0 &  0 \\
            23 & 0  &  0 &  0 &  0 &  0 &  0 &  0 &  1 &  1 \\
            32 & 0  &  0 &  0 &  0 &  0 &  0 &  0 &  1 &  1
        \end{tabular}
    \end{center}
    \caption{The results of $\pdv{E_{ij}}{E_{ab}}$ for a symmetric \EE, rows and columns correspond to the $ij$ and $ab$ tuples and the table values to the partial derivative, clearly there is a difference between the $i=j$ and $i\neq j$ terms.}\label{tab:devtable}
\end{table}

\textbff{So in this work I adopt using $\pdv{E_{ij}}{E_{ab}} = \kappa(\delta_{ai}\delta_{bj} + \delta_{aj}\delta_{bi})$ and $\pdv{E_{ij,k}}{E_{abc}} = \kappa(\delta_{ai}\delta_{bj} + \delta_{aj}\delta_{bi})\delta_{kc}$ and leave $\kappa$ unspecified, hoping that we can then set it to 1 or $\frac{1}{2}$ depending on if $i=j$ in $\fdv{F}{E_{ij}^*}$}

\subsection{Free energies and projection operators}\label{sec:freeenergies}
All of this except the extra term in \cref{eq:fcompdef} are directly from Jack's work, as are both forms for \PP, though only \cref{eq:piex2} was used.
\begin{align}
    F &= \int f_\text{bulk} + f_\text{comp} + f_\text{curv} \dd{V} = F_\text{bulk} + F_\text{comp} + F_\text{curv}\\
    f_\text{bulk} &= \frac{A}{2} E_{ij}E_{ij}^* + \frac{C}{4} (E_{ij}E_{ij}^*)^2 \\
    f_\text{comp} &= b_1^\parallel \Pi_{kl} E_{ij,k}E_{ij,l}^* + b_1^\perp T_{kl} E_{ij,k}E_{ij,l}^* \qq{maybe try adding} b_1^d E_{ij,j}E_{ik,k}^* \qq{later too} \label{eq:fcompdef} \\
    f_\text{curv} &= b_2^\parallel \Pi_{kl}E_{ij,lk}\Pi_{mn}E_{ij,nm}^* + b_2^\perp T_{kl}E_{ij,lk}T_{mn}E_{ij,nm}^* + b_2^{\parallel\perp}(\Pi_{kl}E_{ij,lk}T_{mn}E_{ij,nm}^* + T_{kl}E_{ij,lk}\Pi_{mn}E_{ij,nm}^*)
\end{align}
where
\begin{align}
    \du{\Pi} = \su{N} \su{N} && \text{and} && \du{T} = \du{\delta} - \du{\Pi}
\end{align}
are the projection operators. We need to express these using $\du{E}$ as well, there are 2 options which I quote here
\begin{align}
    \du{\Pi} & = \frac{d-1}{d-2}\qty(\frac{\du{E} \cdot \du{E}^*}{\du{E} \cddot \du{E}^*} - \frac{\du{\delta}}{d(d-1)}) \qq{or} \label{eq:piex1}\\
    \du{\Pi} & = \sqrt{\frac{d-1}{d \du{E} \cddot \du{E}}} \du{E} - \frac{\du{\delta}}{d} \label{eq:piex2} \qq{which has a complex square root}
\end{align}
$\du{T}$ just being calculated from $\du{\Pi}$.

\pagebreak
\section{$F_\text{bulk}$}
\subsection{Using $\kappa$}
\begin{align}
    \fdv{F_\text{bulk}}{E_{ij}^*} &= \frac{A\kappa}{2}(E_{ij} + E_{ji}) + \frac{B\kappa}{4} 2(E_{ab}E_{ab}^*)(E_{ij}+E_{ji}) \\
    &= \kappa(A + B E_{ab}E_{ab}^*) E_{ij} \qq{using symmetry of \EE}
\end{align}

\subsection{Treating the $i=j$ and $i\neq j$ separately, as according to \cref{tab:devtable}}
With \textbff{no sum on $i$}:
\begin{align}
    \fdv{F_\text{bulk}}{E_{ii}^*} &= \frac{A}{2}E_{ii} + \frac{B}{4} 2(E_{ab}E_{ab}^*)E_{ii} \\
    &= (A + B E_{ab}E_{ab}^*)\frac{E_{ii}}{2}
\end{align}
and clearly
\begin{align}
    \fdv{F_\text{bulk}}{E_{ij}^*} &= (A + B E_{ab}E_{ab}^*) E_{ij}
\end{align}

\pagebreak
\section{$F_\text{comp}$ using \cref{eq:piex2}}
(This lead to a symmetric result even without requiring symmetry of \EE, so this should lead to the same result)

To simplify use $s=\sqrt{\frac{d-1}{d}}$ to get
\begin{equation}
    \Pi_{kl} = s \qty(E_{ab}E_{ab})^{-\frac{1}{2}}E_{kl} - \frac{\delta_{kl}}{d} \qq{and recall} f_\text{comp} = (b_1^\parallel - b_1^\perp) \Pi_{kl} E_{ij,k}E_{ij,l}^* + b_1^\perp E_{ij,k}E_{ij,k}^*
\end{equation}
So to get $\fdv{F_\text{comp}}{E_{ij}^*}$, the first term in the Euler-Lagrange equations will be 0 again as only gradients of \EE\ appear directly in the form of $f_\text{comp}$ and \PP\ only has non-conjugated elements of \EE\ appear.
Thus
\begin{align}
    \fdv{F_\text{comp}}{E_{ij}^*} &= -\pp_k \pdv{f_\text{comp}}{E_{ij,k}} = -\pp_k \qty((b_1^\parallel - b_1^\perp) \Pi_{cd} E_{ab,c} \kappa(\ddelta{i}{j}{a}{b})\delta_{kd} + b_1^\perp E_{ab,c}\kappa(\ddelta{i}{j}{a}{b})\delta_{kc}) \\
    &= -\kappa\pp_k \qty((b_1^\parallel - b_1^\perp) \Pi_{ck} (E_{ij,c} + E_{ji,c}) + (E_{ij,k} + b_1^\perp E_{ji,k})) \\
    &= -2\kappa\pp_k \qty((b_1^\parallel - b_1^\perp) \Pi_{ck} E_{ij,c} + b_1^\perp E_{ij,k}) \\
    &= -2\kappa\qty((b_1^\parallel - b_1^\perp) \Pi_{ck,k} E_{ij,c} + (b_1^\parallel - b_1^\perp) \Pi_{ck} E_{ij,ck} + b_1^\perp E_{ij,kk})
\end{align}
So we need
\begin{align}
    \Pi_{ck,k} &= \pp_k \qty( s \qty(E_{ab}E_{ab})^{-\frac{1}{2}}E_{ck} - \frac{\delta_{ck}}{d}) = s \pp_k \qty( \qty(E_{ab}E_{ab})^{-\frac{1}{2}}E_{ck}) \\
    &= s \qty(\qty(E_{ab}E_{ab})^{-\frac{1}{2}}E_{ck,k} - \frac{1}{2}\qty(E_{ab}E_{ab})^{-\frac{3}{2}}E_{ck}2E_{ab}E_{ab,k}) \\
    &= s \qty(\qty(E_{ab}E_{ab})^{-\frac{1}{2}}E_{ck,k} - \qty(E_{ab}E_{ab})^{-\frac{3}{2}}E_{ck}E_{ab}E_{ab,k}) \\
    &= \frac{s}{\sqrt{E_{ab}E_{ab}}} \qty(E_{ck,k} - \frac{E_{ab}E_{ab,k}}{E_{ab}E_{ab}}E_{ck})
\end{align}
so together we get
\begin{align}
    \fdv{F_\text{comp}}{E_{ij}^*} &= -2\kappa\qty(\frac{s(b_1^\parallel - b_1^\perp)}{\sqrt{E_{ab}E_{ab}}} \qty(E_{ck,k} - \frac{E_{ab}E_{ab,k}}{E_{ab}E_{ab}}E_{ck}) E_{ij,c} + (b_1^\parallel - b_1^\perp) E_{ij,ck} + b_1^\perp E_{ij,kk})
\end{align}

\section{The extra term in $F_\text{comp}$}
Should be easy so might as well
\begin{align}
    \fdv{E_{ij}^*}\enspace b_1^d E_{ab,b}E_{ac,c}^* &= b_1^d \pp_d E_{ab,b} \pdv{E_{ac,c}^*}{E_{ij,d}^*} \\
    &= b_1^d \pp_d E_{ab,b} \kappa(\delta_{ia}\delta_{jc}+\delta_{ic}\delta_{ja})\delta_{cd} \\
    &= b_1^d \kappa \pp_d (E_{ib,b}\delta_{jc}\delta_{cd} + E_{jb,b}\delta_{ic}\delta_{cd}) \\
    &= b_1^d \kappa \pp_d (E_{ib,b}\delta_{jd} + E_{jb,b}\delta_{id}) \\
    &= b_1^d \kappa (E_{ib,bd}\delta_{jd} + E_{jb,bd}\delta_{id}) \\
    &= b_1^d \kappa (E_{ib,bj} + E_{jb,bi}) \\
\end{align}
Agress with the result for unconstrained \EE\, except $\kappa$, as expected.

\pagebreak
\section{$F_\text{curv}$ using \cref{eq:piex2}}
Starting from
\begin{align}
    &\Pi_{kl} = s \qty(E_{ab}E_{ab})^{-\frac{1}{2}}E_{kl} - \frac{\delta_{kl}}{d} \qq{and} \\
    &f_\text{curv} = b_2^\parallel \Pi_{kl}E_{ij,lk}\Pi_{mn}E_{ij,nm}^* + b_2^\perp T_{kl}E_{ij,lk}T_{mn}E_{ij,nm}^* + b_2^{\parallel\perp}(\Pi_{kl}E_{ij,lk}T_{mn}E_{ij,nm}^* + T_{kl}E_{ij,lk}\Pi_{mn}E_{ij,nm}^*)
\end{align}
from \cref{sec:freeenergies}, where $s=\sqrt{\frac{d-1}{d}}$.

\subsection{Starting off}
Now to get $\fdv{F_\text{curv}}{E_{ij}^*}$, we can no longer use the same Euler-Lagrange equation as in the previous sections as here our "Lagrangian" depends on second derivatives.
If we use the "integral/delta function" method, integrate by parts and take all boundary terms to zero, it follows that
\begin{align}
    \fdv{\psi(\su{r})}\int f(\psi(\su{r'}), \mgrad \psi(\su{r'}), \mgrad \mgrad \psi(\su{r'})) dV' &= \pdv{f}{\psi} - \mgrad \cdot \pdv{f}{\mgrad \psi} + \mgrad \cdot \mgrad \cdot \pdv{f}{\mgrad \mgrad \psi} \\
    \qq{or} &= \pdv{f}{\psi} - \pp_\alpha \cdot \pdv{f}{(\pp_\alpha \psi)} + \pp_\alpha \pp_\beta \pdv{f}{(\pp_\alpha \pp_\beta \psi)} \\
\end{align}
Adapting to \EE, we get
\begin{equation}
    \fdv{F_\text{curv}}{E_{ij}^*} = \pdv{f_\text{curv}}{E_{ij}^*} - \pp_k \pdv{f_\text{curv}}{E_{ij,k}^*} + \pp_k \pp_l \pdv{f_\text{curv}}{E_{ij,kl}^*}
\end{equation}
where the first two terms are 0, so we just need the last one, but start by simplifying $f_\text{curv}$ to only have \PP
\begin{align}
    f_\text{curv} =&\enspace b_2^\parallel \Pi_{kl}E_{ij,lk}\Pi_{mn}E_{ij,nm}^* + b_2^\perp T_{kl}E_{ij,lk}T_{mn}E_{ij,nm}^* + b_2^{\parallel\perp}(\Pi_{kl}E_{ij,lk}T_{mn}E_{ij,nm}^* + T_{kl}E_{ij,lk}\Pi_{mn}E_{ij,nm}^*) \\
    =&\enspace b_2^\parallel \Pi_{kl}E_{ij,lk}\Pi_{mn}E_{ij,nm}^* + b_2^\perp (\delta_{kl} - \Pi_{kl})E_{ij,lk}(\delta_{mn} - \Pi_{mn})E_{ij,nm}^* \\
    &+\enspace b_2^{\parallel\perp}(\Pi_{kl}E_{ij,lk}(\delta_{mn} - \Pi_{mn})E_{ij,nm}^* + (\delta_{kl} - \Pi_{kl})E_{ij,lk}\Pi_{mn}E_{ij,nm}^*) \nonumber \\
    =&\enspace (b_2^\parallel + b_2^\perp) \Pi_{kl}E_{ij,lk}\Pi_{mn}E_{ij,nm}^* + b_2^\perp (E_{ij,kk}E_{ij,mm}^* - \Pi_{kl}E_{ij,kl}E_{ij,mm}^* - \Pi_{mn}E_{ij,kk}E_{ij,mn}^*)\\
    &+\enspace b_2^{\parallel\perp}(\Pi_{kl}E_{ij,lk}E_{ij,mm}^* + \Pi_{mn}E_{ij,kk}E_{ij,nm}^* - 2\Pi_{kl}E_{ij,lk}\Pi_{mn}E_{ij,nm}^*) \nonumber \\
    =&\enspace (b_2^\parallel + b_2^\perp - 2b_2^{\parallel\perp}) \Pi_{kl}E_{ij,lk}\Pi_{mn}E_{ij,nm}^* \label{eq:fcurv}\\
    &+\enspace (b_2^{\parallel\perp} - b_2^\perp)(\Pi_{kl}E_{ij,lk}E_{ij,mm}^* + \Pi_{mn}E_{ij,kk}E_{ij,nm}^*) + b_2^\perp E_{ij,kk}E_{ij,mm}^* \nonumber
\end{align}

\subsection{Dealing with derivatives of second order derivatives}
Next we want to start taking the derivatives with respect to $E_{ij,kl}^*$, however we have another problem, we have found what this is in \cref{sec:funcder}, the first guess would be $\pdv{E_{ij,kl}}{E_{ab,cd}}=\kappa(\delta_{ai}\delta_{bj} + \delta_{aj}\delta_{bi})\delta_{kc}\delta_{ld}$, however this has the same problem as the $i,j$ indices.
As derivatives commute we have say $E_{ij,12}=E_{ij,21}$, however the mentioned derivative gives 0.
We can symmetrize it as before, introducing another $\kappa$ to fix the $c=d$ or $c\neq d$ issue, however this $\kappa$ is not the same as the other one at that point, it depends on $cd$, not $ij$ so I introduce $\kappa^{ij}=\frac{1}{1+\delta_{ij}}$ (this gives the right values as discussed earlier) and I up the indices upstairs as they sadly make a mess of the summation convention, this has nothing to do with co/contra variance, it's just for me to keep track of them in a clear way without confusing which indices are to be summed over.
With this we have $\pdv{E_{ij,kl}}{E_{ab,cd}}=\kappa^{ab}(\delta_{ai}\delta_{bj} + \delta_{aj}\delta_{bi})\kappa^{cd}(\delta_{kc}\delta_{ld} + \delta_{kd}\delta_{lc})$ with no sums over any variables.

I first take the partial derivative of the terms in \cref{eq:fcurv} separately, start with
\begin{align}
    \pdv{E_{ab,cd}^*} \Pi_{kl}E_{ij,lk}\Pi_{mn}E_{ij,nm}^* &= \kappa^{ab}\kappa^{cd} \Pi_{kl}E_{ij,lk}\Pi_{mn} (\delta_{ia}\delta_{jb}+\delta_{ib}\delta_{ja}) (\delta_{nc}\delta_{md}+\delta_{nd}\delta_{mc}) \\
    &= \kappa^{ab}\kappa^{cd} \Pi_{kl}(E_{ab,lk}+E_{ba,lk})(\Pi_{cd}+\Pi_{dc}) \\
    &= 4\kappa^{ab}\kappa^{cd} \Pi_{kl}E_{ab,lk}\Pi_{cd} \qq{as \EE\ and \PP\ are symmetric}
\end{align}
The next one raises some doubts, but I believe blindly applying the form above works
\begin{align}
    \pdv{E_{ab,cd}^*} \Pi_{kl}E_{ij,lk}E_{ij,mm}^* &= \kappa^{ab}\kappa^{cd} \Pi_{kl}E_{ij,lk} (\delta_{ia}\delta_{jb}+\delta_{ib}\delta_{ja}) (\delta_{mc}\delta_{md}+\delta_{md}\delta_{mc}) \\
    &= 2\kappa^{ab}\kappa^{cd} \Pi_{kl}E_{ab,lk} 2\delta_{cd} \\
    &= 2\kappa^{ab} \Pi_{kl}E_{ab,lk} \delta_{cd} \qq{given the form of $\kappa^{ij}$}
\end{align}
to justify the result, consider $\pdv{\pp_i \pp_j \psi} \nabla^2 \psi = \pdv{\pp_i \pp_j \psi} \sum_k \pp_k \pp_k \psi$ which is clearly 1 if $i=j$ and 0 otherwise, so $\pdv{\pp_i \pp_j \psi} \nabla^2 \psi=\delta_{ij}$, the result above is analogous to this.
Continuing we also have
\begin{align}
    \pdv{E_{ab,cd}^*} \Pi_{mn}E_{ij,kk}E_{ij,nm}^* &= \kappa^{ab}\kappa^{cd} \Pi_{mn}E_{ij,kk} (\delta_{ia}\delta_{jb}+\delta_{ib}\delta_{ja}) (\delta_{nc}\delta_{md}+\delta_{nd}\delta_{mc}) \\
    &= 4\kappa^{ab}\kappa^{cd} \Pi_{cd}E_{ab,kk} \\
    \qq{and}& \nonumber \\
    \pdv{E_{ab,cd}^*} E_{ij,kk}E_{ij,mm}^* &= \kappa^{ab}\kappa^{cd} E_{ij,kk} (\delta_{ia}\delta_{jb}+\delta_{ib}\delta_{ja}) (\delta_{mc}\delta_{md}+\delta_{md}\delta_{mc}) \\
    &= 4\kappa^{ab}\kappa^{cd} E_{ab,kk} \delta_{cd} \\
    &= 2\kappa^{ab} E_{ab,kk} \delta_{cd}
\end{align}

\pagebreak
\subsection{Putting it all together}
\begin{align}
    \pdv{f_\text{curv}}{E_{ab,cd}^*}=&\enspace 4(b_2^\parallel + b_2^\perp - 2b_2^{\parallel\perp})\kappa^{ab}\kappa^{cd} \Pi_{kl}E_{ab,lk}\Pi_{cd} \\
    &+\enspace (b_2^{\parallel\perp} - b_2^\perp)(2\kappa^{ab} \Pi_{kl}E_{ab,lk} \delta_{cd} + 4\kappa^{ab}\kappa^{cd} \Pi_{cd}E_{ab,kk}) + 2b_2^\perp \kappa^{ab} E_{ab,kk} \delta_{cd} \nonumber \\
    &\qq{expecting the $\kappa^{cd}$ to be a problem use the following} \nonumber \\
    \pdv{f_\text{curv}}{E_{ab,cd}^*}=&\enspace 2\kappa^{ab}\delta_{cd} ((b_2^{\parallel\perp}-b_2^\perp)\Pi_{kl}E_{ab,lk}+b_2^\perp E_{ab,kk}) \\
    &+\enspace 4\kappa^{ab}\kappa^{cd} \Pi_{cd} ((b_2^\parallel + b_2^\perp - 2b_2^{\parallel\perp})\Pi_{kl}E_{ab,lk} + (b_2^{\parallel\perp}-b_2^\perp)E_{ab,kk})
\end{align}
so
\begin{align}
    \fdv{F_\text{curv}}{E_{ab}^*} =&\enspace \pp_c\pp_d \pdv{f_\text{curv}}{E_{ab,cd}^*} \\
    =&\enspace 2\kappa^{ab} \pp_c\pp_c ((b_2^{\parallel\perp}-b_2^\perp)\Pi_{kl}E_{ab,lk}+b_2^\perp E_{ab,kk}) \\
    &+\enspace 4\kappa^{ab}\kappa^{cd} \pp_c \pp_d (\Pi_{cd} ((b_2^\parallel + b_2^\perp - 2b_2^{\parallel\perp})\Pi_{kl}E_{ab,lk} + (b_2^{\parallel\perp}-b_2^\perp)E_{ab,kk})) \\
    =&\enspace 2\kappa^{ab} \pp_c ((b_2^{\parallel\perp}-b_2^\perp)(\Pi_{kl,c}E_{ab,lk}+\Pi_{kl}E_{ab,lkc}) + b_2^\perp E_{ab,kkc}) \\
    &+\enspace 4\kappa^{ab}\kappa^{cd} \pp_c (\Pi_{cd} ((b_2^\parallel + b_2^\perp - 2b_2^{\parallel\perp})(\Pi_{kl,d}E_{ab,lk}+\Pi_{kl}E_{ab,lkd}) + (b_2^{\parallel\perp}-b_2^\perp)E_{ab,kkd}) \\
    &\enspace\phantom{+\enspace 4\kappa^{ab}\kappa^{cd} \pp_c} + \Pi_{cd,d} ((b_2^\parallel + b_2^\perp - 2b_2^{\parallel\perp})\Pi_{kl}E_{ab,lk} + (b_2^{\parallel\perp}-b_2^\perp)E_{ab,kk})) \\
    =&\enspace 2\kappa^{ab} ((b_2^{\parallel\perp}-b_2^\perp)(\Pi_{kl,cc}E_{ab,lk}+2\Pi_{kl,c}E_{ab,lkc}+\Pi_{kl}E_{ab,lkcc}) + b_2^\perp E_{ab,kkcc}) \\
    &+\enspace 4\kappa^{ab}\kappa^{cd} (\Pi_{cd} ((b_2^\parallel + b_2^\perp - 2b_2^{\parallel\perp})(\Pi_{kl,dc}E_{ab,lk}+\Pi_{kl,d}E_{ab,lkc}+\Pi_{kl,c}E_{ab,lkd}+\Pi_{kl}E_{ab,lkdc}) \\
    &\enspace\phantom{+\enspace 4\kappa^{ab}\kappa^{cd} (\Pi_{cd} (} + (b_2^{\parallel\perp}-b_2^\perp)E_{ab,kkdc}) \\
    &\enspace\phantom{+\enspace 4\kappa^{ab}\kappa^{cd}} + \Pi_{cd,c} ((b_2^\parallel + b_2^\perp - 2b_2^{\parallel\perp})(\Pi_{kl,d}E_{ab,lk}+\Pi_{kl}E_{ab,lkd}) + (b_2^{\parallel\perp}-b_2^\perp)E_{ab,kkd}) \\
    &\enspace\phantom{+\enspace 4\kappa^{ab}\kappa^{cd}} + \Pi_{cd,d} ((b_2^\parallel + b_2^\perp - 2b_2^{\parallel\perp})(\Pi_{kl,c}E_{ab,lk}+\Pi_{kl}E_{ab,lkc}) + (b_2^{\parallel\perp}-b_2^\perp)E_{ab,kkc}) \\
    &\enspace\phantom{+\enspace 4\kappa^{ab}\kappa^{cd}} + \Pi_{cd,cd} ((b_2^\parallel + b_2^\perp - 2b_2^{\parallel\perp})\Pi_{kl}E_{ab,lk} + (b_2^{\parallel\perp}-b_2^\perp)E_{ab,kk})) \\
\end{align}

\end{document}
