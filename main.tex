\documentclass{article}
\usepackage[utf8]{inputenc}
\usepackage[top=40mm,bottom=40mm,left=30mm,right=30mm]{geometry}
\usepackage{parskip}
\usepackage{amsmath}
\usepackage{physics}
\usepackage{stackengine}
\usepackage{cleveref}
% \usepackage{siunitx}

% We often use underlines here
\newcommand\barbelow[1]{\stackunder[1.2pt]{$#1$}{\rule{1.5ex}{.1ex}}}
\newcommand{\su}[1]{\barbelow{#1}}
\newcommand{\du}[1]{\barbelow{\barbelow{#1}}}

\newcommand{\pp}{\partial}

\newcommand{\YY}[3][j]{E_{#2#1}E_{#3#1}^*}

\def\onedot{$\mathsurround0pt\ldotp$}
\def\cddot{\mathbin{
    \vcenter{\baselineskip1ex \vspace{-0.1ex}\hbox{\onedot}\hbox{\onedot}}
}}

\begin{document}
This is to keep track of deriving the form of $\fdv{F}{E_{ij}^*}$ in terms of $E_{ij}$ only, using
\section{Initial setup}
\begin{align}
    F & = \int f_\text{bulk} + f_\text{comp} + f_\text{curv} \dd{V} = F_\text{bulk} + F_\text{comp} + F_\text{curv}\\
    f_\text{bulk} & = \frac{A}{2} E_{ij}E_{ij}^* + \frac{C}{4} (E_{ij}E_{ij}^*)^2 \\
    f_\text{comp} & = b_1^\parallel \Pi_{kl} E_{ij,k}E_{ij,l}^* + b_1^\perp T_{kl} E_{ij,k}E_{ij,l}^* \\
    f_\text{curv} & = \ldots \\
\end{align}
where
\begin{align}
    \du{\Pi} = \su{N} \su{N} && \text{and} && \du{T} = \du{\delta} - \du{\Pi}
\end{align}
are the projection operators. We need to express these using $\du{E}$ as well, there are 2 options which I quote here
\begin{align}
    \du{\Pi} & = \frac{d-1}{d-2}\qty(\frac{\du{E} \cdot \du{E}^*}{\du{E} \cddot \du{E}^*} - \frac{\du{\delta}}{d(d-1)}) \qq{or} \label{eq:piex1}\\
    \du{\Pi} & = \sqrt{\frac{d-1}{d \du{E} \cddot \du{E}}} \du{E} - \frac{\du{\delta}}{d} \qq{which has a complex square root}
\end{align}
$\du{T}$ just being calculated from $\du{\Pi}$.

\section{$\fdv{F_\text{comp}}{E_{ij}^*}$ using \cref{eq:piex1} for $\du{\Pi}$ and the Euler-Lagrange derivative}
As $\fdv{F_\text{bulk}}{E_{ij}^*}$ has been calculated before and it is the easier one, I move straight to the compression contribution. There the function in the integral only depends on the first derivatives of $\du{E}$ so we can use the usual Euler-Lagrange expression for calculating the functional.
\begin{align}
    \fdv{\phi}\int f(\phi, \grad \phi) \dd{V} &= \pdv{f}{\phi} - \grad \cdot \pdv{f}{(\grad \phi)} \qq{or for us} \\
    \fdv{E_{ij}^*}\int f_\text{comp}(\du{\Pi}(\du{E}, \du{E}^*), \grad \du{E}, \grad \du{E}^*) \dd{V} &= \pdv{f_\text{comp}}{E_{ij}^*} - \grad \cdot \pdv{f_\text{comp}}{(\grad E_{ij}^*)} \label{eq:eulag}
\end{align}
Here I use \cref{eq:piex1} as the expression for $\du{\Pi}$ furhter using $p = \frac{d-1}{d-2}$ and $q = \frac{1}{d(d-1)}$ as shorthands.
I repeat the equations here for convenience
\begin{equation}
    \Pi_{kl} & = p\qty(\frac{\YY{k}{l}}{\YY{i}{i}} - q \delta_{kl}) \qq{and} f_\text{comp} & = b_1^\parallel \Pi_{kl} E_{ij,k}E_{ij,l}^* + b_1^\perp (\delta_{kl} - \Pi_{kl}) E_{ij,k}E_{ij,l}^*
\end{equation}

\subsection{The first term}
Noticing that the only dependence that $f_\text{comp}$ only depends on $\du{E}$ through $\du{\Pi}$ lets first get
\begin{align}
    \pdv{\Pi_{kl}}{E_{ij}^*} &= p \qty(\frac{E_{kc}\delta_{li}\delta_{cj}}{\YY[b]{a}{a}} - \frac{\YY[c]{k}{l}}{(\YY[b]{a}{a})^2}E_{ab}\delta_{ai}\delta_{bj}) \\
    &= p \qty(\frac{E_{kj}\delta_{li}}{\YY[b]{a}{a}} - \frac{\YY[c]{k}{l}E_{ij}}{(\YY[b]{a}{a})^2}) \label{eq:asym1}
\end{align}
\noindent\fbox{\parbox{\textwidth}{
    Okay so here it got a little sketchy - the problem is that I always do $\pdv{E_{ij}^*}{E_{ab}^*} = \delta_{ia}\delta_{jb}$, but I'm not sure this is compatible with $\du{E}$ being symmetric - then we should really have $\pdv{E_{ij}^*}{E_{ab}^*} = \delta_{ia}\delta_{jb} + \delta_{ib}\delta_{ja}$ right? HOWEVER, I'm pretty sure we want to treat the different components of $\du{E}$ as independent, just "coincidentally" having the same values everywhere (and so also gradients), which is maintained by the Lagrange multiplier.
}}
then we get
\begin{align}
    \pdv{f_\text{comp}}{E_{ij}^*} &= b_1^\parallel \pdv{\Pi_{kl}}{E_{ij}^*}E_{\alpha\beta,k}E_{\alpha\beta,l}^* - b_1^\perp \pdv{\Pi_{kl}}{E_{ij}^*}E_{\alpha\beta,k}E_{\alpha\beta,l}^* \\
    &= (b_1^\parallel - b_1^\perp) \pdv{\Pi_{kl}}{E_{ij}^*}E_{\alpha\beta,k}E_{\alpha\beta,l}^* \\
    &= p(b_1^\parallel - b_1^\perp) \qty(\frac{E_{kj}\delta_{li}}{\YY[b]{a}{a}} - \frac{\YY[c]{k}{l}E_{ij}}{(\YY[b]{a}{a})^2}) E_{\alpha\beta,k}E_{\alpha\beta,l}^* \\
    &= p(b_1^\parallel - b_1^\perp) \qty(\frac{E_{kj}E_{\alpha\beta,k}E_{\alpha\beta,i}^*}{\YY[b]{a}{a}} - \frac{\YY[c]{k}{l}E_{ij} E_{\alpha\beta,k}E_{\alpha\beta,l}^*}{(\YY[b]{a}{a})^2}) \\
    &= \frac{p(b_1^\parallel - b_1^\perp)}{(\YY[b]{a}{a})^2} \qty(\YY[b]{a}{a}E_{kj}E_{\alpha\beta,k}E_{\alpha\beta,i}^* - \YY[c]{k}{l}E_{ij} E_{\alpha\beta,k}E_{\alpha\beta,l}^*) \\
    &= \frac{p(b_1^\parallel - b_1^\perp)}{Y_{aa}^2} \qty(Y_{aa}E_{kj}E_{\alpha\beta,k}E_{\alpha\beta,i}^* - Y_{kl}E_{ij} E_{\alpha\beta,k}E_{\alpha\beta,l}^*) \qq{with} Y_{ij} = \YY[k]{i}{j}
\end{align}
which is suspicious to me as it's the first thing I encountered that is not symmetric.
I also tried to first substitute in to $\du{\Pi}$ and then do the derivative and I got the same result, I think it is correct.
Coming back to \cref{eq:asym1} the first term there really encapsulates the problem, on the LHS both $\du{\Pi}$ and $\du{E}$ are symmetric but the first term on the RHS is not invariant under switching $i$ and $j$ or $k$ and $l$.
I am currently hoping that maybe it cancels with a part of the second term

\subsection{The second term}
Let's start working on the second term of \cref{eq:eulag}, firstly as the projection operators only depend on the $\du{E}$ and not on any of its derivatives we have
\begin{align}
    \pdv{f_\text{comp}}{(E_{ab,c}^*)} &= b_1^\parallel \Pi_{kl} E_{ij,k}\delta_{ia}\delta_{jb}\delta_{lc} + b_1^\perp T_{kl} E_{ij,k}\delta_{ia}\delta_{jb}\delta_{lc} \\
    &= b_1^\parallel \Pi_{kc} E_{ab,k} + b_1^\perp \qty(\delta_{kc} - \Pi_{kc}) E_{ab,k}
\end{align}
As the next step is to take a divergence of the above, lets first calculate
\begin{align}
    \pp_\alpha \Pi_{\beta\gamma} &= p \mathbin{\pp_\alpha} \frac{E_{\beta j}E_{\gamma j}^*}{E_{ij}E_{ij}^*} \\
    &= p \qty(\frac{E_{\beta j, \alpha}E_{\gamma j}^*}{E_{ij}E_{ij}^*} + \frac{E_{\beta j}E_{\gamma j, \alpha}^*}{E_{ij}E_{ij}^*} - \frac{E_{\beta j}E_{\gamma j}^*}{(E_{ij}E_{ij}^*)^2}\qty(E_{ij,\alpha}E_{ij}^* + E_{ij}E_{ij,\alpha}^*) ) && \text{or} \\
    \pp_\alpha \Pi_{\beta\gamma} &= p \qty(\frac{Y_{\beta\gamma,\alpha}}{Y_{ii}} - \frac{Y_{\beta\gamma} Y_{ii,\alpha}}{Y_{ii}^2}) \qq{using} Y_{ij} = E_{ik}E_{jk}^*
\end{align}
which leads to
\begin{align}
    \pp_c \pdv{f_\text{comp}}{(E_{ab,c}^*)} &= \qty(b_1^\parallel \Pi_{kc,c} E_{ab,k} - b_1^\perp \Pi_{kc,c} E_{ab,k}) + \qty(b_1^\parallel \Pi_{kc} E_{ab,kc} + b_1^\perp \qty(\delta_{kc} - \Pi_{kc}) E_{ab,kc}) \\
    &= \qty(b_1^\parallel - b_1^\perp) \Pi_{kc,c} E_{ab,k} + \qty(b_1^\perp E_{ab,cc} + \qty(b_1^\parallel - b_1^\perp) \Pi_{kc} E_{ab,kc})
\end{align}
bracketing to separate the first and second order derivatives. Turns out we only need
\begin{equation}
    \pp_\beta \Pi_{\alpha\beta} &= p \qty(\frac{Y_{\alpha\beta,\beta}}{Y_{ii}} - \frac{Y_{\alpha\beta} Y_{ii,\beta}}{Y_{ii}^2})
\end{equation}
leading to
\begin{align}
    \pp_c \pdv{f_\text{comp}}{(E_{ab,c}^*)} &= p\qty(b_1^\parallel - b_1^\perp) \qty(\frac{Y_{kc,c}}{Y_{ii}} - \frac{Y_{kc} Y_{ii,c}}{Y_{ii}^2}) E_{ab,k} + \qty(b_1^\perp E_{ab,cc} + \qty(b_1^\parallel - b_1^\perp) \Pi_{kc} E_{ab,kc})
\end{align}

\subsection{Combining the terms}
\begin{align}
    \fdv{f_\text{comp}}{E_{ij}^*} =&\enspace
\end{align}

\end{document}
