\documentclass[11pt]{article}
\usepackage[top=40mm,bottom=40mm,left=20mm,right=20mm]{geometry}

\usepackage[utf8]{inputenc}
\usepackage{parskip}
\usepackage{amsmath}
\usepackage{physics}
\usepackage{stackengine}
\usepackage{float}
\usepackage{cleveref}
\usepackage{siunitx}

\allowdisplaybreaks

\newcommand\textbff[1]{\textbf{\boldmath #1}}
\newcommand\circled[1]{\raisebox{.5pt}{\textcircled{\raisebox{-.9pt} {#1}}}
    }

% We often use underlines here
\newcommand\barbelow[1]{\stackunder[1.2pt]{$#1$}{\rule{1.5ex}{.1ex}}}
\newcommand{\su}[1]{\barbelow{#1}}
\newcommand{\du}[1]{\barbelow{\barbelow{#1}}}

\newcommand{\pp}{\partial}

\newcommand{\YY}[3][j]{E_{#2#1}E_{#3#1}^*}

\def\onedot{$\mathsurround0pt\ldotp$}
\def\cddot{\mathbin{
    \vcenter{\baselineskip1ex \vspace{-0.1ex}\hbox{\onedot}\hbox{\onedot}}
}}


\begin{document}
\begin{center}
    \LARGE
    \textbf{Exploring various constraints and their consequences}
\end{center}
\vspace{1em}
\section{Normality, symmetry and tracelessness}\label{sec:nosytr}
\begin{align}
    \du{E} &= \du{U}^\dagger \begin{pmatrix}
        \lambda_1 & 0 & 0 \\
        0 & \lambda_2 & 0 \\
        0 & 0 & - \lambda_1 - \lambda_2
    \end{pmatrix} \du{U} \qq{where $\du{U}$ is unitary} \\
    &= (\lambda_1+\lambda_2)\du{U}^\dagger \begin{pmatrix}
        \frac{\lambda_1}{\lambda_1+\lambda_2} & 0 & 0 \\
        0 & \frac{\lambda_2}{\lambda_1+\lambda_2} & 0 \\
        0 & 0 & -1
    \end{pmatrix} \du{U} \\
    &= (\lambda_1+\lambda_2)\du{U}^\dagger \qty(\begin{pmatrix}
        \frac{\lambda_1}{\lambda_1+\lambda_2}+1 & 0 & 0 \\
        0 & \frac{\lambda_2}{\lambda_1+\lambda_2}+1 & 0 \\
        0 & 0 & 0
    \end{pmatrix} - \du{\delta}) \du{U} \\
    \implies E_{ij} &= (\lambda_1+\lambda_2)\qty(\alpha_1 U_{1i}^*U_{1j} + \alpha_2 U_{2i}^*U_{2j} - \delta_{ij}) \qq{with} \alpha_i = \frac{\lambda_i}{\lambda_1+\lambda_2} + 1 \\
    \qq{or} E_{ij} &= \beta_1 U_{1i}^*U_{1j} + \beta_2 U_{2i}^*U_{2j} - (\lambda_1+\lambda_2)\delta_{ij} \qq{with} \beta_i = 2\lambda_i + \lambda_\text{the other}
\end{align}
Imposing symmetry seems a bit tricky, but try it generally as
\begin{equation}
    \beta_1 U_{1i}^*U_{1j} + \beta_2 U_{2i}^*U_{2j} = \beta_1 U_{1j}^*U_{1i} + \beta_2 U_{2j}^*U_{2i}
\end{equation}
then by a very physicsy argument, the $\beta$ are free to be anything (note that the $\lambda \rightarrow \beta$ transformation is invertible, so we can treat the $\beta$ as our degrees of freedom), and the symmetry must still hold, thus for $k\in\{i,j\}$
\begin{equation}
    U_{ki}^*U_{kj} = U_{kj}^*U_{ki} \qq{no sum on $k$}
\end{equation}
which leads to each $\su{U_k}$ being of the same form as in NormalitySymmetryEigenvalues, essentially a real unit vector (with it being possibly constrained to half of the options due to parity symmetry of the dyadic products above).
Note that they still have to be mutually orthogonal.
This leads to
\begin{align}
    \du{E} &= \beta_1 \su{U_1}\su{U_1} + \beta_2 \su{U_2}\su{U_2} - (\lambda_1+\lambda_2)\du{\delta} \\
    &\qq{recognising the $\beta$ are a more natural choice} \\
    \du{E} &= \beta_1 \su{U_1}\su{U_1} + \beta_2 \su{U_2}\su{U_2} - \frac{\beta_1 + \beta_2}{3}\du{\delta} \\
    &= \beta_1 (\su{U_1}\su{U_1} - \frac{\du{\delta}}{3}) + \beta_2 (\su{U_2}\su{U_2} - \frac{\du{\delta}}{3})
\end{align}

Also, to summarize for possible future reference relationship of $\beta$s and $\lambda$s:
\begin{equation}
    \begin{pmatrix} \beta_1 \\ \beta_2 \end{pmatrix} = \begin{pmatrix} 2 & 1 \\ 1 & 2 \end{pmatrix} \begin{pmatrix} \lambda_1 \\ \lambda_2 \end{pmatrix} \Leftrightarrow \begin{pmatrix} \lambda_1 \\ \lambda_2 \end{pmatrix} = \frac{1}{3}\begin{pmatrix} 2 & -1 \\ -1 & 2 \end{pmatrix} \begin{pmatrix} \beta_1 \\ \beta_2 \end{pmatrix}
\end{equation}

\pagebreak
\textbff{Thus requiring normality, symmetry and tracelessness of \EE\ is equivalent to requiring it to have the form of $\psi_1(\su{N}\su{N} - \frac{\du{\delta}}{3}) + \psi_2(\su{M}\su{M} - \frac{\du{\delta}}{3})$ for $\psi_1$, $\psi_2$ being independent complex numbers and $\su{N}$ and $\su{M}$ being orthogonal real unit vectors (optionally constrained to half of the space), a form very reminiscent of the biaxial form of the nematic $\du{Q}$}

Overall this gives 2+2+3=7 degrees of freedom, 2 for each complex number and 3 for the 2 unit vectors.
Note that the result does not have a single complex phase!
If we were to force that they do have the same phase, we go down to 6 dof (the same number of dof for a single-complex-phase, symmetric and traceless \EE), and it has a questionable physics interpretation -- we are fixing the $\phi$s of the 2 order parameters, while keeping independent $|\psi|$ -- it does not seem to imply uniaxiality.

Uniaxiality would occur if $\beta_2 = 0 \Leftrightarrow \lambda_1 = -2\lambda_2$ which gives the recogniseable, doubly-degenerate eigenvalues $-2\lambda, \lambda, \lambda$

\section{Summary of considered constraints}
In any case we require symmetry (in practice achieved through symmetrising the functional derivatives) and tracelessness (enforced through a relatively simple Lagrange multiplier).
Then there are essentially 2 paths, the more theoretically clear one is requiring normality leading to the results in \cref{sec:nosytr} and then considering whether or how to force uniaxiality.
The other, used before is requiring a single complex phase (aka $\du{E} = e^{i\phi}T_{ij}$, shorthanded as SCP), this naturally implies normality.

Path 1 gives a fully biaxial \EE\ with 7 dof.

Path 2 implies normality and thus all of path 1, but it is easy to count the resulting \EE\ has 1 dof for the phase, then 6 - 1 = 5 dof for the symmetric and traceless real matrix.
So overall 6 dof, thus we get that SCP is a stronger constraint than only normality, with the minimum extra constraint being that the eigenvalues of \EE\ have to have the same phase, but independent magnitudes, being somewhat partially biaxial.

So in summary, the minumum constraints for a biaxial \EE\ are normality, symmetry and traceleness. And neither SCP nor any other set of constraints explored so far are sufficient to guarantee uniaxiality, which would only have 4 dof (2 for $\psi$ and 2 for $\su{N}$).



\end{document}
