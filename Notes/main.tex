\documentclass[11pt]{article}
\usepackage[top=40mm,bottom=40mm,left=20mm,right=20mm]{geometry}

\usepackage[utf8]{inputenc}
\usepackage{parskip}
\usepackage{amsmath}
\usepackage{physics}
\usepackage{stackengine}
\usepackage{float}
\usepackage{cleveref}
\usepackage{siunitx}

\allowdisplaybreaks

\newcommand\textbff[1]{\textbf{\boldmath #1}}
\newcommand\circled[1]{\raisebox{.5pt}{\textcircled{\raisebox{-.9pt} {#1}}}
    }

% We often use underlines here
\newcommand\barbelow[1]{\stackunder[1.2pt]{$#1$}{\rule{1.5ex}{.1ex}}}
\newcommand{\su}[1]{\barbelow{#1}}
\newcommand{\du}[1]{\barbelow{\barbelow{#1}}}

\newcommand{\pp}{\partial}

\newcommand{\YY}[3][j]{E_{#2#1}E_{#3#1}^*}

\def\onedot{$\mathsurround0pt\ldotp$}
\def\cddot{\mathbin{
    \vcenter{\baselineskip1ex \vspace{-0.1ex}\hbox{\onedot}\hbox{\onedot}}
}}


\begin{document}
\subsubsection*{5th Dec}
Dealing with the code, fixed the initialization (which fixed the tracelessness again).

Doing runs where every step is printed -- the energies look nice without any oscillations and the constraints are performing surprisingly well!

Still trying to figure out if the uniaxial stuff is working alright, currently I'm seeing that the reconstructed n is not of unit norm, it can be ${\sim}1.2$ or so -- this is from a fully randomized starting point, but phi=0.
This means the reconstructed pi has trace not 1! this may be important -- possible correction, do $\du{\Pi}' = \frac{\du{\Pi}}{\Tr(\du{\Pi})}$ -- this should be equivalent to normalizing the underlying $\su{n}$ but not affecting the \PP\ in any other way (thus might be a stable and sensible way to correct even invalid \PP).

\subsubsection*{6th Dec}
Continuing the same as yesterday but also want to get non-dimensialization theory and code done today!

Another thing I didn't mention so far but is worth noting, if phi is non-zero the simulation almost immediately goes to all nans -- problem and it must be something numerical, but I think I will only look into this once it has the updated logic -- both non-dimensialization and projection operators, or at least one of them.
This happens even with a fixed initial phi of 0.000001, the nans happen within 2 steps.
\end{document}
